\documentclass[11pt]{article}
\usepackage{physics} 
\usepackage{siunitx} 
\usepackage{enumerate} 
\usepackage{pgfplots}
\usepackage{graphicx}
\usepackage{pgfplotstable}
\usepackage{tikz,pgfplots}
\usepackage{amsmath} 
\usepackage{xcolor}

\usepackage[]{lineno}  \linenumbers
\setlength\linenumbersep{3pt}
	
\usepackage{geometry}
\geometry{total={170mm,240mm}, left=20mm, top=20mm}

\begin{document}

\title{Maps that erase arbitrary components of the density matrix of a $n$-qubits system} %Title should be concise and to the point  
\author{J. A. de Le\'on} 


\date{\today}  

\maketitle


According to \cite{nielsen_chuang_2011}, an arbitrary density matrix on $n$
qubits can be expanded as
\begin{align}
	\rho = \frac{1}{2^n}\sum _{\vec{v}} \Tr 
                  \qty( \sigma _{v_1} \otimes \sigma_{v_2} \otimes 
               \cdots \otimes \sigma_{v_n} \rho) \sigma _{v_1} \otimes 
             \sigma_{v_2} \otimes \ldots \otimes \sigma_{v_n},
	\label{rho}
\end{align}
where the sum is over vectors $\vec{v}=\qty(v_1,\cdots, v_n)$ with entries
$v_i$ chosen from the set $\{0,1,2,3\}$. We label the coefficients $\Tr \qty(
\sigma _{v_1} \otimes \sigma_{v_2} \otimes \cdots \otimes \sigma_{v_n}
\rho )$ as $r_{v_1, v_2,\ldots, v_n}$ to shorten the notation. 

The kind of maps that act on density matrices of the form \eqref{rho} that
we're interested in are those which leave invariant or `erase' the coefficients
in $\rho$ (i.e. $r'_{v_1, v_2,\ldots, v_n}=r_{v_1, v_2,\ldots, v_n}$ or
$r'_{v_1, v_2,\ldots, v_n}=0$, where the primed $r$'s refer to the coefficients
in the transformed density matrix $\rho '$). 

So far, with a numerical method we've characterized all 1 and 2 qubit maps, whereas 
for 3 qubits system we have studied maps that leave invariant 1, 2, 3 and 4
coefficients in $\rho$. Our results exhibit the following features:
\begin{itemize}
\item Valid quantum channels are only found for maps that leave invariant a
power of two number of coefficients. We have not found an explanation for that.
\item 
% There's not a relation 1-on-1 for the number of possible maps that leave
% invariant a certain number of coefficients with the number valid quantum
% channels found. It seems that 
Not only the number of coefficients to leave
invariant is taken into account but actually which coefficients are left
invariant too. 
\item Valid maps for $n$ qubits must be valid for subsets of qubits. For example, 
% \item The results are recursive by increasing the number of qubits in the
% system, i.e. the valid quantum channels for 2 qubits have to obey the valid
% quantum channels for 1 qubit. In more detail, 
the 2-qubits quantum-channels
that leave invariant any of the coefficients of the form $r_{v_1,0}$ or
$r_{0,v_2}$ have to obey the valid quantum channels found for 1 qubit. 
This induces nice rules that are not accessible (we think) by considering 
a single $2^n$-level quantum system. {\color{red} [Nota]
No entiendo la utilidad de este punto. Salvo la regla que nos permite ver si el canal es un producto tensorial de canales (o no), no se a que otras reglas se refieren. Claramente un producto tensorial de canales es tal que los 1s internos coinciden con los 1s en las orillas. Si se especificará mas este punto, propongo borrarlo, creo que mete ruido.}
\end{itemize}
We are currently exploring equivalence relations (using particle permutations and local unitaries) to reduce the ridiculous
number of possible maps, in order to achieve a decent computing time using numerical methods (this number increases as $2^{n^2-1}$). 

We tried to understand if the kind of maps of our interest are a particular
case of the `maps constant on axes' defined by M.  Nathanson and M.B. Ruskai
in Equation (3) of \cite{nathanson2007pauli}. We are still unable to 
answer this question. 
For 1 qubit, we showed that the classical-quantum
channels $\Psi ^{\text{QC}}$, together with the unit element, form the set of
valid quantum channels of our results. Nevertheless, for 2 qubits the $\Phi
^{\text{QC}}$'s are only a subset of our results, and we still cannot prove or
disprove that the remaining quantum channels of our results can be constructed
as convex combinations of the $\Psi ^{\text{QC}}$s.

The analysis I made goes as follows. Considering the case of 2 qubits: Equation (8) of \cite{nathanson2007pauli} is an
equivalent form of \eqref{rho} with operators $W$ as the tensor products of the Pauli matrices, therefore the coefficients
$v_{Jj}$ are our $r_{v_1,v_2}$, excluding $r_{0,0}$. As the authors state, the action of the channels that they describe in Equation (2)
is to take $v_{Jj}\to \qty(s+t_J)v_{Jj}$. Every $v_{Jj}$ corresponds to one of the fifteen coefficients that may be left invariant.
For every $J$, there are three $v_{Jj}$s. Thus, each $\Psi ^{\text{QC}}_J$ leaves invariant three different
coefficients and $r_{0,0}$. So it seems that one can only leave invariant 1, 4, 7, 10, 13 or 16 coefficients with a convex
combination of the classical-quantum channels $\Psi ^{\text{QC}}_J$. {\color{red} [Nota] creo que vale la pena explicar despacio esta ultima linea. De donde salen esos numeros?, es lo único que no me queda claro de este parrafo. Nos reunimos o crees que puedes explicar esto un poco mas pausado?, José Alfredo. Quizá valga la pena construir un ejemplo, digamos, el que deja invariante $10$ componentes.}

\bibliographystyle{unsrt}
\bibliography{references}


\end{document}
