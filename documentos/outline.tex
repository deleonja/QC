\documentclass[11pt]{article}
\usepackage{physics} 
\usepackage{siunitx} 
\usepackage{enumerate} 
\usepackage{pgfplots}
\usepackage{graphicx}
\usepackage{pgfplotstable}
\usepackage{tikz,pgfplots}
\usepackage{amsmath} 
\usepackage{xcolor}

\usepackage[]{lineno}  \linenumbers
\setlength\linenumbersep{3pt}
	
\usepackage{geometry}
\geometry{total={170mm,240mm}, left=20mm, top=20mm}

\begin{document}

\title{Projective maps on a system of $n$ qubits} 
%Title should be concise and to the point  
\author{J. A. de Le\'on} 


\date{\today}  

\maketitle


According to \cite{nielsen_chuang_2011}, an arbitrary density matrix on $n$  
qubits can be expanded as
\begin{align}
	\rho = \frac{1}{2^n}\sum _{\vec{v}} \Tr 
                  \qty( \sigma _{v_1} \otimes \sigma_{v_2} \otimes 
               \cdots \otimes \sigma_{v_n} \rho) \sigma _{v_1} \otimes 
             \sigma_{v_2} \otimes \ldots \otimes \sigma_{v_n},
	\label{rho}
\end{align}
where the sum is over vectors $\vec{v}=\qty(v_1,\cdots, v_n)$ with entries
$v_i$ chosen from the set $\{0,1,2,3\}$. We label the coefficients $\Tr \qty(
\sigma _{v_1} \otimes \sigma_{v_2} \otimes \cdots \otimes \sigma_{v_n}
\rho )$ as $r_{v_1, v_2,\ldots, v_n}$ to shorten the notation. 

The kind of maps that act on density matrices of the form \eqref{rho} that
we're interested in are those which leave invariant or erase the 
coefficients
in $\rho$ (i.e. $r'_{v_1, v_2,\ldots, v_n}=r_{v_1, v_2,\ldots, v_n}$ or
$r'_{v_1, v_2,\ldots, v_n}=0$, where the primed $r$'s refer to the 
coefficients in the transformed density matrix $\rho '$). 

So far, with a numerical method we've characterized all 1 and 2 qubit maps, 
whereas for 3 qubits system we have studied maps that leave invariant 1, 2, 
3 and 4 coefficients in $\rho$. Our results exhibit the following features:
\begin{itemize}
	\item Valid quantum channels are only found for maps that leave invariant
				a power of 2 number of coefficients, meaning it is a necessary
				condition but not sufficient for characterizing this kind of 
				maps. We have not found an explanation for that.
				
	\item 
% There's not a relation 1-on-1 for the number of possible maps that leave
% invariant a certain number of coefficients with the number valid quantum
% channels found. It seems that 

%				Not only the number of coefficients to leave
%				invariant is taken into account but actually which coefficients 
%				are left invariant too. 

				As a consequence of the previous item, not only the number of 
				coefficients to leave invariant is taken into account but actually
				which coefficientes are left invariant too.
				
	\item Valid maps for $n$ qubits must be valid for subsets of qubits. For example, 
% \item The results are recursive by increasing the number of qubits in the
% system, i.e. the valid quantum channels for 2 qubits have to obey the valid
% quantum channels for 1 qubit. In more detail, 
the 2-qubits quantum-channels
that leave invariant any of the coefficients of the form $r_{v_1,0}$ or
$r_{0,v_2}$ have to obey the valid quantum channels found for 1 qubit. 
This induces nice rules that are not accessible (we think) by considering 
a single $2^n$-level quantum system. {\color{red} [Nota]
No entiendo la utilidad de este punto. Salvo la regla que nos permite ver si el canal es un producto tensorial de canales (o no), no se a que otras reglas se refieren. Claramente un producto tensorial de canales es tal que los 1s internos coinciden con los 1s en las orillas. Si se especificará mas este punto, propongo borrarlo, creo que mete ruido.}
\end{itemize}

The results for 2 qubits where classified in the following manner: 
the resulting density matrices from applying valid quantum channels
to an arbitrary density matrix where arranged in `equivalence classes' 
such that elements in a class are connected by tranposition or 
permutations of the rows or columns 1, 2 and 3. The equivalence classes do not 
depend solely on the number of components left invariant in $\rho$. 
For example, four equivalence classes are found for quantum channels
that leave 4 components invariant.

We would like to
\begin{itemize}
	\item search for simple rules to generate the elements 
				of the equivalence classes,
	\item verify if there are unitary transformations that connect elements of
				different classes.
\end{itemize}
\bibliographystyle{unsrt}
\bibliography{references}


\end{document}
