\documentclass[11pt,dvipsnames]{article} % {{{

\usepackage{geometry}
\geometry{total={170mm,240mm}, left=20mm, top=20mm}

\usepackage[utf8]{inputenc}

\usepackage{physics} 
\usepackage{siunitx} 
\usepackage{enumerate} 
\usepackage{pgfplots}
\usepackage{graphicx}
\usepackage{pgfplotstable}
\usepackage{tikz,pgfplots}
\usepackage{amsmath} 
\usepackage{xcolor}
\usepackage{float}
\usepackage{amsfonts}
\newcommand{\pce}{PCE}
\usepackage{dsfont}
%\usepackage{bbold}


\newcommand{\fref}[1]{fig.~\ref{#1}}  \newcommand{\tref}[1]{table~\ref{#1}}
\newcommand{\Fref}[1]{Fig.~\ref{#1}}  \newcommand{\Tref}[1]{Table~\ref{#1}}

\newcommand{\h}[1]{\colorbox{Yellow}{#1}}
\newcommand{\1}{\mathds{1}}
\newcommand{\R}{\mathcal{R}}

\usepackage[]{lineno}  \linenumbers
\setlength\linenumbersep{3pt}
	

\usepackage{fancybox}
\usepackage{colortbl}
\usepackage{amsbsy}
\usepackage[draft,inline,nomargin]{fixme} \fxsetup{theme=color}
\FXRegisterAuthor{cp}{acp}{\color{blue}CP}
\FXRegisterAuthor{ja}{aja}{\color{RedViolet}JA}
\FXRegisterAuthor{dd}{ddg}{\color{red}DD}
% }}}
\begin{document}
% Titulo y otros {{{
\title{Projective maps on a system of $n$ qubits} 
%Title should be concise and to the point  
\author{J. A. de León\\ \small{in collaboration with C. Pineda, D. Dávalos, A. Fonseca}}


\date{\today}  

\maketitle
% }}}
% Intro {{{
According to \cite{bengtsson_zyczkowski_2017}, an arbitrary density matrix on $n$  
qubits can be written as
\begin{align}
  \rho = \frac{1}{2^n}\1 + \sum _{i=1}^{2^{2n}-1}\tau _i\sigma _i,
	\label{rho}
\end{align}
where the $\sigma _i$s are tensor products of Pauli 
matrices, which together the identity, form a basis in the space of $2^n\times 2^n$ matrices. Thus the 
$\tau_i$s are the projections of $\rho$ onto each element of this basis, using the Hilbert-Schmidt inner product.

In this work, we are interested in studying the set of maps that erase
some of the components $\tau_i$ in $\rho$, and characterize the subset of maps
that are quantum channels.
\janote{Introduje la notación a continuación. Se ponía engorroso
no tener una notación así en la sección de mapeos de Ruskai} 
\h{Let us call this subset Pauli component erasing
channels -\pce{} channels-.}
%We will refer to the coefficients in both terms of 
%\eqref{rho} as the components in $\rho$ \ddnote{No se que quieres decir con esta frase JA, si no es nada importante, quitemosla}. 
For 1 qubit the picture of our problem is easy to understand given that 
$\tau_i$ are the components of a vector in the Bloch ball, the so-called
Bloch vector. As an example, let us consider a map that erases any of the 
components of the Bloch vector. Geometrically, this map collapses 
the Bloch ball into a disk. This operation is not completely positive, and therefore is not a quantum channel.  

So far, evaluating complete positivity numerically, we have obtained all 1 and 2 \pce{} channels. Whereas for $3$ qubits systems we have numerically analyzed only maps that leave 
invariant 1, 2, 3, 4, and 64 components in $\rho$. Nonetheless, we have strong
indications that only maps that leave invariant 8 components are 
needed to analyze in order to find the complete set of 3-qubit PCE 
channels.
% }}}
\section*{Results} % {{{
In \fref{fig:pictorial-rep-rho} we introduce a pictorial representation 
for the identity map acting on a density matrix for systems 
of 1, 2 and 3 qubits,
respectively. It will be helpful to visualize the maps we are studying 
making use of this tool. We'll consider any little square or cube in blank
as a component erased in $\rho$ by the map. 
\janote{Agregué la siguiente línea por tu observación sobre los colores 
que está antes de la Fig. 3.}
Red squares correspond to components of Bloch vectors of one-particle reduced density matrices,
blue squares to correlations between any pair of qubits, and
green squares to correlations between all qubits in the system for
the 3-qubits case.
In the following subsections we present the PCE channels for
1 and 2 qubits, and preliminary results for 3 qubits.
\begin{figure}[H] % {{{
	\centering
	\hfill \hfill
	\includegraphics[height=2.5cm]
	{img/tablero-1q}
	\hfill
	\includegraphics[width=2.5cm]
	{img/rho2q(2)}
	\hfill 
	\includegraphics[width=2.5cm]
	{img/rho-3q}
	\hfill \hfill
	\caption{From left to right the identity map acting on
  an arbitrary density matrix of 1, 2 and 3 qubits, respectively. 
	Red squares correspond to components of local Bloch vectors,
	blue squares to correlations between any pair of qubits and
	green squares to correlations between all qubits in the system, for
	the 3-qubits case. }
	\label{fig:pictorial-rep-rho}
\end{figure} % }}}
\subsection*{1 qubit} % {{{
In \fref{fig:1q-ccs} we present the PCE channels for 1 qubit.
The first and last patterns represent identity 
and the total depolarizing channel. 
Patterns in between represent 
bit-phase flip, phase-flip and bit-flip channels (from left to right), 
all of them for $p=1/2$.	
These three flip channels are equivalent via permutation
of the components $\tau_i$ of the Bloch vector in \eqref{rho}.
\begin{figure}[H]% {{{
	\centering
	\includegraphics[width=5cm]
	{img/1q-CCs.png}
	\caption{
\cpnote{Mejora el caption. } \janote{Listo.}
1-qubit PCE channels. From left to right patterns represent  
the following channels: identity,
bit-phase flip, phase flip, bit flip (the last three for 
$p=0.5$), and completely depolarizing.}
	\label{fig:1q-ccs}
\end{figure} % }}}
% }}}
\subsection*{2 qubits} % {{{
2-qubits PCE channels have been classified in equivalence classes 
(as shown from \fref{fig:2q-c1} to \fref{fig:2q-c16}), such
that elements in a class are connected by
\begin{enumerate}
	\item Particle swaps: Qubits in the system can be interchanged. Therefore
	if a map is a PCE channel for a certain arrangement 
	of qubits, then
	there are equivalent PCE channels acting on the system for all possible
	arrangements of qubits. In the pattern-picture this can be
	interpreted as transpositions.
	\item Permutations of individual components: The state of an individual
	qubit may be rotated in such a way that components of its local Bloch
	vector are swapped via permutations. PCE channels are equivalent
	to these permutations.
	In the pattern-picture this
	is interpreted as permutations of rows or columns 1, 2 and 3.
\cpnote{Explica el por que de eso. En los dos casos, mas bien 
di como clasificamos, desde un punto de vista fisico, y luego a que traduice graficamente
eso. primero lo importante y luego el detalle. }
\janote{Listo.}
\end{enumerate}
\cpnote{No recuerdo que tan pronto mencionas lo de los colores azules y eso, pero seria 
bueno explicarlo pronto y ser consistente con eso desde la primera figura}

\begin{figure}[H] % {{{
	\centering
  \includegraphics[height=1.2cm]
	{img/C16.png}
	\caption{C${}^{1}$}
	\label{fig:2q-c1}
\end{figure} % }}}

\begin{figure}[H] % {{{
	\begin{minipage}[c]{0.5\textwidth}
		\centering
	  \includegraphics[width=.9\textwidth]
		{img/C12.png}
		\vspace{1.2cm}
		\caption{C${}_1^2$}
	\end{minipage}\hfill
	\begin{minipage}[c]{0.5\textwidth}
		\centering
	  \includegraphics[width=.9\textwidth]
		{img/C22.png}
		\caption{C${}_2^2$}
	\end{minipage}
\end{figure} % }}}

\begin{figure}[H] % {{{
	\begin{minipage}[c]{0.5\textwidth}
		\centering
	  \includegraphics[width=.9\textwidth]
		{img/C14.png}
		\vspace{1.2cm}
		\caption{C${}_1^4$}
	\end{minipage}\hfill
	\begin{minipage}[c]{0.5\textwidth}
		\centering
	  \includegraphics[width=.9\textwidth]
		{img/C24.png}
		\caption{C${}_2^4$}
	\end{minipage}\vfill
\begin{minipage}[c]{0.5\textwidth}
		\centering
	  \includegraphics[width=.9\textwidth]
		{img/C34.png}
		\caption{C${}_3^4$}
	\end{minipage}\hfill
	\begin{minipage}[c]{0.5\textwidth}
		\centering
	  \includegraphics[width=.9\textwidth]
		{img/C44.png}
		\vspace{2.5cm}
		\caption{C${}_4^4$}
		\label{fig:2q-4c-2}
	\end{minipage}
\end{figure} % }}}

\begin{figure}[H] % {{{
	\begin{minipage}[c]{0.5\textwidth}
		\centering
	  \includegraphics[width=.9\textwidth]
		{img/C18.png}
		\vspace{1.2cm}
		\caption{C${}_1^8$}
	\end{minipage}\hfill
	\begin{minipage}[c]{0.5\textwidth}
		\centering
	  \includegraphics[width=.9\textwidth]
		{img/C28.png}
		\caption{C${}_2^8$}
	\end{minipage}
\end{figure} % }}}

\begin{figure}[H] % {{{
	\centering
  \includegraphics[height=1.2cm]
	{img/C0.png}
	\caption{C${}^{16}$}
	\label{fig:2q-c16}
\end{figure} % }}}

In general, our results exhibit the following features:
\begin{itemize}
\item Only a power-of-2 number of components in $\rho$ 
may be left invariant by PCE 
channels. However, not only the number of components in $\rho$ 
to leave invariant determines complete positivity since
not all maps that leave $2^{k}$ components invariant in $\rho$
are quantum channels. For example, the pattern below corresponds 
to a map that is a potential element of C${}_2^4$ (\Fref{fig:2q-4c-2}).
Nevertheless, complete positivity is not satisfied 
and then the map is not a quantum channel. 
\cpnote{Menciona con que clase y que figura
tendría que ocntrastar. También en la siguiente parte sobra el 
ambiente figure, pues ni le pones caption ni quieres que flote. Se lo quité}. 
\janote{Listo.}
% \begin{figure}[H]
% 	\centering
% 	\includegraphics[height=2cm]{img/not-cc}
% \end{figure}
\begin{center}
	\includegraphics[height=2cm]{img/not-cc}
\end{center}

\item 
The ratio of PCE channels and all possible maps
that erase components in $\rho$ according to number of qubits 
and number of components
invariant are shown in \fref{fig:CCs-by-components}.
\cpnote{ En la figura, creo que los colores sobre y
distraen. Tambien sería padre poner cuantos de cada clase hay. Por ejemplo
poner 1/1 que quiere decir que el unico posible es canal, o 63/128 queriendo
decir que 63 de los 128 son canales.} \janote{Listo.} \ddnote{Me parece que este punto esta incompleto, tampoco entiendo completamente la figura.}

%Two numbers
%of same color correspond to quantum channels that we suspect have a 1:1 
%correspondence. In the next item we extend this discussion.

\begin{figure}[H]% {{{
	\centering
	\begin{tabular}{>{$n=$}l<{\hfill}*{12}{c}}
1 &&&&&\colorbox{Apricot}{1/1}&3/3&\colorbox{Apricot}{1/1}&&&&&5\\
2 &&&&\colorbox{CadetBlue}{1/1}&\colorbox{Cyan}{15/15}&35/455&\colorbox{Cyan}{15/6435}&\colorbox{CadetBlue}{1/1}&&&&67\\
3 &&&\colorbox{SpringGreen}{1/1}&\colorbox{RedOrange}{63/63}&\colorbox{Yellow}{651/39711}&?/$6\times 10^8$&\colorbox{Yellow}{¿651?/$1\times 10^{14}$}&
\colorbox{RedOrange}{¿63?/$9\times 10^{17}$}&\colorbox{SpringGreen}{1/1}&&&?
\end{tabular}
\caption{First column shows the number of qubits in the system.  In the second
column each position correspond to the number of components invariant ($2^0,
2^1, \ldots, 2^{2n}$) and the numbers shown are the ratios of PCE channels
and all possible maps according to the number of components invariant.  Finally, third column
specifies the total number of PCE channels for a $n$-qubit system.}
\label{fig:CCs-by-components}
\end{figure} % }}}

\item Empirical observations of results for 2 qubits 
led us to some rules that 
patterns showed from \fref{fig:2q-c1} to \fref{fig:2q-c16} obey:
\begin{enumerate}
\item If a component with indices $ij$ is left invariant, then two options are
allowed: \textit{a)} both components with indices $i0$ and $0j$ are left
invariant too, or \textit{b)} both components with indices $i0$ and $0j$ are
erased.
Let us take a look at the next example. 
%\cpnote{Cambié el hfill y los newlines
%por un center. Cuando lo veas quita este comentario}
% \newline
% \hfill
\begin{center}
\includegraphics[width=2cm]{img/ex-2q2c-empiricalRule}
\end{center}
% \hfill \hfill
% \newline
The component with indices 21 in the pattern is left invariant,
then the only options allowed are: \textit{a)} components with indices 01 and
20 are erased, as in this pattern above, or \textit{b)} components 
with indices 01 and 20 are left invariant too, 
as in the following pattern 
\begin{center}
% \newline
% \hfill
\includegraphics[width=2cm]{img/ex-2q4c-empiricalRule},
% \hfill \hfill 
% \newline
\end{center}
which is in fact another PCE channel. This rule can be derived using the complete positive condition over the following pattern:
\begin{equation}
\begin{array}{|c|c|c|c|}
\hline
1 & a_1 & 0 & 0 \\ 
\hline
a_2 & b & 0 & 0 \\ 
\hline
0 & 0 & 0 & 0 \\ 
\hline
0 & 0 & 0 & 0\\
\hline
\end{array}
\end{equation}
% \janote{Fin.}

\item Considering only components in the correlation matrix (blue squares): if a component
with indices $ij$ is left invariant and the previous rule is obeyed, then,
in the correlation matrix, remaining components on row $i$ and column $j$ are
erased, and the rest of the components are left invariant by a PCE channel. 
Let's go back to the last pattern in the previous example. This rule 
ensures that components in the correlation matrix outside of row 2 and 
column 1 may be left invariant by a PCE channel, i.e.
\begin{center}
\includegraphics[width=2cm]{img/ex-2q8c-empiricalRule},
\end{center}
which is another PCE channel.
\end{enumerate}	
% Actually, this two simple rules allow us to connect two elements of classes
% C${}^2_k$ and C${}^8_k$. \cpnote{Creo que se pueden hacer más conexiones, no?}
% \janote{Es cierto, corrijo:}
Actually, this two simple rules allow us to relate
different equivalence classes. In the previous example we were able to connect 
an element of C${}_2^2$ to one of C${}_3^4$, to another one of
C${}_2^8$.

% \janote{Agrego el siguiente item porque no le había dado la suficiente
% importancia unos items arriba (en el de la fig. 13)}
\item \textit{(Rainbow hypothesis)} PCE channels
that leave $2^k$ components invariant and $2^{2n-k}$ have a
1:1 correspondence. That is to say, two numbers of the same color 
in \fref{fig:CCs-by-components} correspond to PCE
channels that have a 1:1 correspondence. 

\item The action of a PCE channel on every subsystem must be another
PCE channel. 
% \cpnote{Esto también hay que explicar que efecto tiene en los diagramas}
% \janote{\h{Inicio}}
This can be seen from the patterns in \fref{fig:2q-c1} to \fref{fig:2q-c16}.
Recall that red squares represent components of local Bloch vectors 
and blue squares represent correlations shared between qubits. 
Then, it may be noted in the first column and first row of every
pattern there is a 1-qubit PCE quantum channel of the form of \fref{fig:1q-ccs}.
% \janote{\h{Fin}}
\end{itemize}
% }}}
\subsection*{3 qubits} % {{{

Recall that for 3 qubits system we have numerically analyzed only maps that
leave invariant 1, 2, 3, 4, and 64 components in $\rho$. Therefore, results
presented in this section are preliminary. Maps that are PCE channels follow the same
features of 2-qubits system. Not all 3-qubits PCE channels 
will be shown as in
the previous section, but only one element of every equivalence class found.
Elements in equivalence classes are connected via particle swaps and
permutation of individual components.

\subsubsection*{1-invariant-component maps} % {{{
The only map that leaves invariant 1 component in $\rho$ is the completely
depolarizing channel, and it is trivially a PCE channel.
\begin{figure}[H]
	\includegraphics[height=4cm]{img/3q-1c}
	\caption{Completely depolarizing channel for 3 qubits.}
	\label{fig:QC-3q-1c}
\end{figure}
% }}}
\subsubsection*{2-invariant-components maps}% {{{
All 63 maps that leave invariant 2 components in $\rho$ are 
PCE channels. It's important to mention that in the 2-qubits case all 
2-invariant-components maps are PCE channels, too. We can distinguish
3 equivalence classes: quantum channels that leave invariant
\begin{enumerate}
	\item any component of a local Bloch vector,
	\item any correlation between any pair of qubits,
	\item any correlation between all qubits in the system.
\end{enumerate}
\begin{figure}[H]
	\centering
	\hfill \hfill
	\includegraphics[height=4cm]{img/3q-2c-1}
	\hfill
	\includegraphics[height=4cm]{img/3q-2c-2}
	\hfill
	\includegraphics[height=4cm]{img/3q-2c-3}
	\hfill \hfill
	\caption{3-qubits 2-components-invariant PCE channels. 
	One element of each of the 3 equivalence classes 
	(from left to right): leaves invariant one component of any local
	Bloch vector, one correlation between any pair or qubits, and 
	one correlation between all qubits in the system.}
	\label{fig:QC-3q-2c}
\end{figure}

% }}}

\cpnote{Estudiates los que dejan invariantes 3 elementos?}
\janote{Sí. Agrego la siguiente sección:}
\subsubsection*{3-invariants-component maps}
No PCE channels were found in the set of all maps that leave
3 components invariant in $\rho$.

\subsubsection*{4-invariant-components maps} % {{{
There are 39,711 maps that leave invariant 4 components in $\rho$, 651
are PCE channels and may be classified in 10 equivalence classes.
One arbitrary element in each class is shown in \fref{fig:QC-3q-4c}. 
We understand how to infer 5 out of the 10 equivalence classes 
from 1 and 2-qubits PCE channels, as it will be discussed.

On top of \fref{fig:QC-3q-4c} the first 4 elements
(from left to right) correspond to PCE channels that are separable
in a 1 or 2 qutbis PCE channel acting on any subsystem 
and a completely depolarzing channel acting on the rest of the system.
The fifth and last element on top of the figure may be understood invoking    
an extension for 3 qubits of one of the empirical rules presented 
for 2 qubits. The extensions of those rules will be discussed at
the end of this section. 
PCE channels belonging to equivalence classes of bottom elements in 
the figure are such that we still cannot infer them with our 
previous results.

\begin{figure}[H]
	\centering
	\hfill \hfill
	\includegraphics[height=3cm]{img/3q-4c-si-1}
	\hfill
	\includegraphics[height=3cm]{img/3q-4c-si-2}
	\hfill
	\includegraphics[height=3cm]{img/3q-4c-si-3}
	\hfill
	\includegraphics[height=3cm]{img/3q-4c-si-4}
	\hfill
	\includegraphics[height=3cm]{img/3q-4c-si-5}
	\hfill
	\vfill
	\hfill
	\includegraphics[height=3cm]{img/3q-4c-no-1}
	\hfill
	\includegraphics[height=3cm]{img/3q-4c-no-2}
	\hfill
	\includegraphics[height=3cm]{img/3q-4c-no-3}
	\hfill
	\includegraphics[height=3cm]{img/3q-4c-no-4}
	\hfill
	\includegraphics[height=3cm]{img/3q-4c-no-5}
	\hfill \hfill
	\caption{3-qubits 4-components-invariant PCE channels. 
	One element of every of the 10 equivalence classes.}
	\label{fig:QC-3q-4c}
\end{figure}
% }}}

The empirical rules already presented for 2 qubits can be 
formulated for 3 qubits as follows:
\begin{itemize}
	\item A component with indices $ijk$ (all different from zero) is
	left invariant by a PCE channel if and only if 
	one of the cases is followed:
	\begin{enumerate}
		\item Components $ij0$, $0jk$, $i0k$, $i00$, $0j0$, and
		$00k$ are erased.
		\item Components $i00$ and $0jk$ are also left invariant.
		\item Components $0j0$ and $i0k$ are also left invariant.
		\item Components $00k$ and $ij0$ are also left invariant.
		\item Components $ij0$, $0jk$, $i0k$, $i00$, $0j0$, and
		$00k$ are also left invariant.
	\end{enumerate}
	It can be noted that the element in the right upper corner of 
	\fref{fig:QC-3q-4c} is understood as one of the cases 2-4 of this item.
	\item Considering only components that correspond to correlations between
	3 qubits in the system (green cubes): if a little cube 
	with indices $ijk$ (all different from zero) is left invariant 
	and any of the cases of the previous rule is obeyed, 
	then the remaining components on columns $i$, $j$, and $k$ are
	erased and the rest of components in the correlation tensor 
	are left invariant by a PCE channel. 
\end{itemize}
Let us consider an example to make use of both empirical rules. If we 
begin with an element of the same equivalence class as the second
element on top of \fref{fig:QC-3q-4c}, then we make use of one of 
the cases 2-4 of the first empirical rule for 3 qubits and get the PCE channel below.
\begin{center}
	\includegraphics[width=4cm]{img/3q-8c-je}.
\end{center}
Finally, we make use of the second rule to left invariant only components
corresponding to correlations between all qubits in the system (green cubes)
and get another PCE channel (pattern below).
\begin{center}
	\includegraphics[width=4cm]{img/3q-8c}
\end{center}
What we have done is start with a PCE channel that leaves 4 invariant
components, make our way trough a 8-invariant-components PCE channel and 
finally arrive at a 16-invariant-components PCE channel.

In principle, the rainbow hypothesis ensures the correspondence between  
the channels we have found and channels that leave invariant 16, 32 and 64
components. The empirical rules for 3 qubits provide one way to \textit{travel}
from a $k$-components-invariant quantum channel to a
$(2n-k)$-components-invariant channel.
It is sufficient to found one element in an 
equivalence class because the rest of the elements are found 
by particle swaps and permutation of individual components.
% }}}
\cpnote{Sería importante discutir como se comportan todas las hipotesis que hiciste 
para dos qubits en el caso de 3 qubits}
\janote{De acuerdo. Lo agregué después de la fig. 16}


% }}}
\section*{Are this channels a subset of Pauli diagonal channels constant on axes?} % {{{
We explored if PCE channels are contained within the set of Pauli diagonal
channels constant on axes \cite{nathanson2007pauli}. Let us call them
$\mathcal{R}$. For 2 qubits, we concluded that Kraus rank $4$ 
PCE channels are the only candidates to
be in $\mathcal{R}$, in fact we are convinced that all of them are in, but are
still working in the proof. 
%\cpnote{David is convinced, the rest of us have no idea} 
To see this let us analyze the action of maps in $\R$ 
on arbitrary density matrices
\begin{align}
	\rho \longmapsto \frac{1}{d}\qty[\1 + \sum _{J=1}^{d+1} \lambda_J
	\sum _{j=1}^{d-1}	v_{Jj}W_J^j],
	\label{eq:ruskai-action}
\end{align}
where $\lambda_J$ are the eigenvalues of a given map, and operators $W_J$ are unitaries defined as
\begin{equation}
	W_J = \sum _{k=1}^{d}\omega^k \dyad{\psi_k^J}{\psi_k^J}, 
\end{equation}
where sets of vectors $\qty{\ket{\psi_k^J}}_k$ with $J=1,\dots, 1+d$ are mutually unbiased basis (MUB) over $\mathbb{C}^d$, and $\omega = e^{2\pi i/d}$. There are $d^2-1$ unitaries generated with the powers of the $W_J$s, i.e.
$\qty{W_J^m}_{m=1,\ldots,d,J=1,\ldots,d+1}$. They, together the identity matrix, form a basis in the space of $d\times d$ matrices. \ddnote{JA, he visto que dices que una base de unitarias es solo una base para matrices unitarias, no es cierto.} 
%\fxnote{Por fas
%corrige la argumentacion a partir de la modificacion a la ecuacion} 

%\fxnote{JA, ojo
%que los $W_J$ generan un grupo ciclico, mas no generadores de MUBs, uno
%construye los $W_J$ usando MUBs!. Por fas mejor mete la definicion de $W_J$.}
 

To examine if PCE channels are contained in $\R$ let us consider
the rank of both maps and discuss how only a subset of PCE channels
could be in $\R$. Our results show that PCE channels have 
a power-of-2 rank. On the other hand, from \eqref{eq:ruskai-action} it
can be seen that the rank of a map in $\R$ is 1 plus 
the number of all $\lambda_J$ different from zero. Notice that eigenvalues $\lambda_J$ have multiplicity $d-1$, for each $J$. Consequently, the allowed matrix ranks 
of maps in $\R$ are $1 + k(d-1)$, where $k = 1,2,\ldots,d+1$. It follows
that a \textit{necessary} condition for maps in $\R$ and PCE channels to have the 
same rank is that
\begin{align}
	\hspace{2cm}2^j&=1 + k(2^{2n}-1),	\hspace{1.5cm} j=0,1,\ldots,2n
\end{align}
is always true for some $k\in \mathbb{Z^+}$, and $n$ the number 
of qubits. If we take $n=2$ and $j=3$ 
(2 qubits, $2^3$ components invariant)
\begin{align}
	k&= \frac{2^3-1}{2^{2\qty(2)}-1}=\frac{7}{15}.
\end{align}
Therefore the rank of a PCE channel may not be the same of any map in $\R$.
Then, there are PCE channels that cannot be in $\R$, like
all 8-components-invariant PCE channels, as shown in the previous example. 


%\cpnote{Creo que eso ya lo teníamos claro. Sería bueno que discutieramos pues 
%me gustaría dar mas seguridades. De hecho tu argumento me parece una prueba, pero quizá
%lo tengamos que refinar. Planeemos una reunion pronto con el resto del combo para ver esto.
%Obvio, creo, si hay intersección, pero no son los mismos.}
% }}}
\section*{To-do} % {{{
In order to fully understand our results and generalize this kind of maps 
for $n$ qubits we propose the following:
\begin{enumerate}
\item Investigate the Kraus operator representation of this quantum channels.
\item Investigate the Schmidt spectrum of the Choi matrix.
\item Investigate the Jamiolkowski isomorphism to find an equivalence between
CP and the empirical rules listed previously.
\item Use our current results to propose an efficient way to do numerical
analysis to find 3-qubit quantum channels that leave 8 components invariant.
\end{enumerate}
% }}}
\bibliographystyle{unsrt}
\bibliography{references}
\vfill

\end{document}
