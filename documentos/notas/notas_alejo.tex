\section{Alejo's ideas}
% 
\subsection{Eigenvalues of the Choi matrix}
% 
Due to the fact that the Pauli matrices and the identity form a basis of the Hilber-Shmidt space, the state of a qubit may be written as
% 
\begin{equation}
 \hat{\rho}=\frac{1}{2}\sum_{j,k=0}^1r_{jk}\hat{v}_{jk},
\end{equation}
%
where $\hat{v}_{00}=\dsone$, $\hat{v}_{01}=\hat{\sigma}_x$, $\hat{v}_{10}=i\hat{\sigma}_y$ and $\hat{v}_{11}=\hat{\sigma}_z$.
% 
Rewriting $\hat{\rho}$ in ``vector" notation, we have
%
\begin{equation}
 \vec{\rho}=\frac{1}{2}\sum_{jk=0}^1r_{jk}\vec{v}_{jk},
\end{equation}
%
The action of a quantum operation $\hat{M}$ (in its matrix form), capable of modifying each one of the coefficients $r_{jk}$ independently may be written as
%
\begin{equation}
 \hat{M}\vec{\rho}=\frac{1}{2}\sum_{jk=0}^1\tau_{jk}r_{jk}\vec{v}_{jk}.
\end{equation}
%
In our particular case (PCE operations), $\tau_{jk}=\{0,1\}$. 

In order to determine whether any specific operation, characterized by the set of coefficients $\{\tau_{jk}\}$ corresponds to a quantum channel, we must ensure that the eigenvalues associated to the Choi matrix are non-negative. For this, let write $\hat{M}$ in the vector Pauli basis
%
\begin{equation}
 \hat{M}=\sum_{jk=0}^1\tau_{jk}~\vec{v}_{jk}\vec{v}_{jk}^\intercal.
 \label{EQA1}
\end{equation}
%
On the other hand, it is easy to show that the elements of the computational and Pauli bases are related by $\vec{u}=\hat{T}\vec{v}$
\begin{equation}
\vec{u}=\begin{pmatrix}
\hat{u}_{00} \\ \hat{u}_{01} \\ \hat{u}_{10} \\ \hat{u}_{11}
\end{pmatrix}=\frac{1}{2}\begin{pmatrix}
1 & 0 & 0 & 1\\
0 & 1 & 1 & 0\\
0 & 1 & -1 & 0\\
1 & 0 & 0 & -1
\end{pmatrix} \begin{pmatrix}
\hat{v}_{00} \\ \hat{v}_{01} \\ \hat{v}_{10} \\ \hat{v}_{11}
\end{pmatrix},
\end{equation}
% 
where $\hat{u}_{jk}=\dyad{j}{k}$. The inverse relation reads $\vec{v}=\hat{S}\vec{u}$
\begin{equation}
\vec{v}=\begin{pmatrix}
\hat{v}_{00} \\ \hat{v}_{01} \\ \hat{v}_{10} \\ \hat{v}_{11}
\end{pmatrix}=\begin{pmatrix}
1 & 0 & 0 & 1\\
0 & 1 & 1 & 0\\
0 & 1 & -1 & 0\\
1 & 0 & 0 & -1
\end{pmatrix} \begin{pmatrix}
\hat{u}_{00} \\ \hat{u}_{01} \\ \hat{u}_{10} \\ \hat{u}_{11}
\end{pmatrix}.
\end{equation}
% 
Thus, for this ``indexing choice" of the elements of the bases, the transformation matrices satisfy $\hat{S}=2\hat{T}$. Then one can relate the bases elements in a compact form as
% 
\begin{equation}
 \vec{v}_{jk}=\sum_{mn=0}^1S_{mn}^{(jk)}\vec{u}_{mn},
\end{equation}
% 
by substituting this in Eq. \ref{EQA1} we get $\hat{M}$ in the computational basis
% 
\begin{align}
 \hat{M}=&\sum_{jk=0}^1\tau_{jk}~\left(\sum_{mn=0}^1S_{mn}^{(jk)}\vec{u}_{mn} \right) \left(\sum_{pq=0}^1S_{jk}^{(pq)}\vec{u}^\intercal_{pq} \right),\\
 =&\sum_{mnpq=0}^1\left(\sum_{jk=0}^1 S_{mn}^{(jk)}S_{jk}^{(pq)}  \tau_{jk} \right) \vec{u}_{mn}\vec{u}^\intercal_{pq}.\\
 &\sum_{mnpq=0}^1 M_{\substack{mn\\pq}} ~\vec{u}_{mn}\vec{u}^\intercal_{pq}.
\end{align}
% 
In order to obtain the Choi matrix $\hat{M}^c$, we must perform the reshuffling operation on $\hat{M}$, i.e. $M_{\substack{mn\\pq}} \to M_{\substack{mp\\nq}}$. In this way, $\hat{M}^c$ holds

\begin{equation}
\hat{M}^c=\sum_{mnpq=0}^1\left(\sum_{jk=0}^1 S_{mp}^{(jk)}S_{jk}^{(nq)}  \tau_{jk} \right) \vec{u}_{mn}\vec{u}^\intercal_{pq}.
\end{equation}
%
Back to the Pauli basis, we have
%
\begin{equation}
 \hat{M}^c=\sum_{\mu\nu\alpha\beta=0}^1\left\{\sum_{jk=0}^1\left[\sum_{mnpq=0}^1S_{mp}^{(jk)}S_{jk}^{(nq)}T_{\mu\nu}^{(mn)}T_{pq}^{(\alpha\beta)}\right]\tau_{jk}\right\}\vec{v}_{\mu\nu}\vec{v}_{\alpha\beta}^\intercal.
\end{equation}
%
By taking into account the relation between the transformation matrices $T_{\mu\nu}^{(mn)}=\frac{1}{2}S_{\mu\nu}^{(mn)}$, $\hat{M}^c$ reads
%
\begin{equation}
 \hat{M}^c=\frac{1}{4}\sum_{\mu\nu\alpha\beta=0}^1\left\{\sum_{jk=0}^1\left[\sum_{mnpq=0}^1S_{mp}^{(jk)}S_{jk}^{(nq)}S_{\mu\nu}^{(mn)}S_{pq}^{(\alpha\beta)}\right]\tau_{jk}\right\}\vec{v}_{\mu\nu}\vec{v}_{\alpha\beta}^\intercal.
\end{equation}
%
By noticing that $S_{mn}^{(jk)}=(-1)^m\delta_{jm}\delta_{kn}+\delta_{j,m\oplus 1}\delta_{k,n\oplus 1}$, and after several steps it is possible to show that 
% 
\begin{equation}
\frac{1}{4}\sum_{jk=0}^1\left[\sum_{mnpq=0}^1S_{mp}^{(jk)}S_{jk}^{(nq)}S_{\mu\nu}^{(mn)}S_{pq}^{(\alpha\beta)}\right]\tau_{jk}=\frac{1}{2}\delta_{\nu\beta}\delta_{\mu\alpha}\sum_{jk=0}^1(-1)^{j\mu\oplus k\nu}\tau_{jk},
\label{EQAL2}
\end{equation}
% 
after substituting, the Choi matrix reduces to
%
\begin{align}
 \hat{M}^c=&\frac{1}{2}\sum_{\mu\nu\alpha\beta=0}^1\left\{\delta_{\nu\beta}\delta_{\mu\alpha}\sum_{jk=0}^1(-1)^{j\mu\oplus k\nu}\tau_{jk}\right\}\vec{v}_{\mu\nu}\vec{v}_{\alpha\beta}^\intercal,\\
 =&\frac{1}{2}\sum_{\mu\nu=0}^1\left\{\sum_{jk=0}^1(-1)^{j\mu\oplus k\nu}\tau_{jk}\right\}\vec{v}_{\mu\nu}\vec{v}_{\mu\nu}^\intercal.
\end{align}
%
As it can be seen, this is the diagonal form of the Choi matrix. Thus, its 4 eigenvalues may be written in a compact form as
%
\begin{equation}
\lambda_{\mu\nu}=\frac{1}{2}\sum_{jk=0}^1(-1)^{j\mu\oplus k\nu}\tau_{jk},
\end{equation}
%
where $\mu,\nu=\{0,1\}$. Note in addition that this is in agreement with results already known in literature.

By following the same steps we can obtain the generalization to the $N$-qubits case. First of all note that $\hat{M}$ may be written as
%
\begin{equation}
 \hat{M}=\sum_{\vec{k}\vec{l}}\tau_{\vec{k}\vec{l}}~\vec{v}_{\vec{k}\vec{l}}\vec{v}_{\vec{k}\vec{l}}^\intercal,
%  \label{EQA1}
\end{equation}
% 
where
% 
\begin{equation}
 \vec{v}_{\vec{k}\vec{l}}=\vec{v}_{k_1l_1}\otimes\dots\vec{v}_{k_Nl_N}=\bigotimes_{j=1}^N\vec{v}_{k_jl_j}.
%  \label{EQA1}
\end{equation}
% 
Let write $\hat{M}$ in the $N$-qubits computational basis
% 
\begin{align}
 \hat{M}=&\sum_{\vec{k}\vec{l}\vec{p}\vec{q}}\left\{\sum_{\vec{m}\vec{n}} \left[ \prod_{i=1}^N S_{k_il_i}^{(m_in_i)}S_{m_in_i}^{(p_iq_i)} \right] \tau_{\vec{m}\vec{n}} \right\} \bigotimes_{j=1}^N\vec{u}_{k_jl_j}\vec{u}_{p_jq_j}^\intercal,\\
 =&\sum_{\vec{k}\vec{l}\vec{p}\vec{q}} M_{\vec{k}\vec{l}\vec{p}\vec{q}} ~\bigotimes_{j=1}^N\vec{u}_{k_jl_j}\vec{u}_{p_jq_j}^\intercal.
\end{align}
% 
After reshuffling ($M_{\vec{k}\vec{l}\vec{p}\vec{q}}\to M_{\vec{k}\vec{p}\vec{l}\vec{q}}$) we can get the associated Choi matrix. Back to the Pauli basis, $\hat{M}^c$ holds
%
\begin{equation}
 \hat{M}^c=\frac{1}{4^N}\sum_{\vec{\mu}\vec{\nu}\vec{\alpha}\vec{\beta}}\left\{\sum_{\vec{m}\vec{n}}\left[\prod_{i=1}^N\left(\sum_{k_il_ip_iq_i=0}^1 S_{k_ip_i}^{(m_in_i)} S_{m_in_i}^{(l_iq_i)}S_{\mu_i\nu_i}^{(k_il_i)}S_{p_iq_i}^{(\alpha_i\beta_i)}\right)\right]\tau_{\vec{m}\vec{n}}\right\} \bigotimes_{j=1}^N\vec{v}_{\mu_j\nu_j}\vec{v}_{\alpha_j\beta_j}^\intercal.
\end{equation}
% 
Using the result in Eq. \ref{EQAL2} and after some steps it can be shown that
%
\begin{equation}
 \hat{M}^c=\sum_{\vec{\mu}\vec{\nu}}\left\{\frac{1}{2^N}\sum_{\vec{\alpha}\vec{\beta}}(-1)^{\vec{\mu}\cdot\vec{\alpha}\oplus \vec{\nu}\cdot\vec{\beta}}\tau_{\vec{\alpha}\vec{\beta}}\right\}\vec{v}_{\vec{\mu}\vec{\nu}}\vec{v}_{\vec{\mu}\vec{\nu}}^\intercal,
\end{equation}
% 
which is also the diagonal form of $\hat{M}^c$. In this way, the $4^N$ eigenvalues may be written compactly as
% 
\begin{equation}
 \lambda_{\vec{\mu}\vec{\nu}}=\frac{1}{2^N}\sum_{\vec{\alpha}\vec{\beta}}(-1)^{\vec{\mu}\cdot\vec{\alpha}\oplus \vec{\nu}\cdot\vec{\beta}}\tau_{\vec{\alpha}\vec{\beta}}.
\end{equation}
% 
From this we can evaluate whether the set of coefficients $\tau_{\vec{\alpha}\vec{\beta}}$ satisfy the condition to the specific operation be considered a quantum channel, i.e. $\lambda_{\vec{\mu}\vec{\nu}}\geq 0$, for all $\vec{\mu},\vec{\nu}$.
 
% 
% 
% 
% 
% 
% 
% Usando o mesmo método foi possível generalizar o resultado para o caso de um canal envolvendo um número arbitrário de qubits $N$, obtendo a seguinte expressão
% %
% \begin{equation}
% \lambda_{\vec{\mu}\vec{\nu}}=\frac{1}{2^N}\sum_{\vec{j}\vec{k}}(-1)^{\vec{j}\cdot\vec{\mu}\oplus \vec{k}\cdot\vec{\nu}}\tau_{\vec{j}\vec{k}},
% \end{equation}
% %
% onde o símbolo``$\oplus$" representa soma módulo 2, a dimensão dos vetores envolvidos é $N$, e cada uma das componentes pode atingir os valores $0$ ou $1$. Vale a pena destacar que esta expressão concorda com resultados numéricos obtidos previamente.
% 
% O resultado anteriormente mencionado possui grande importância pois além do fato de não ter sido previamente reportado, fornece uma ferramenta analitica para avaliar as condições que devem satisfazer os coeficientes de uma operação quântica $\tau_{\vec{j}\vec{k}}$, para ser considerados canales quânticos válidos. No caso particular de canais tipo PCE, dentro das características mais relevantes extraídas da expressão dos autovalores que vale a pena mencionar, temos que somente uma quantidade $2^m$ de coeficientes $\tau_{\vec{j}\vec{k}}$ pode ser não nulo.
% 
% 
% 
% 
% 
% 
% 
% 
% 
% 
% 
% 
% 
% % It is possible to show that the eigenvalues of the Choi matrix associated to a Pauli channel of $N$ qubits may be written in a compact form as
% % 
% \begin{equation}
%  \lambda_{\vec{\mu}\vec{\nu}}=\sum_{\vec{m}\vec{n}}(-1)^{\vec{\mu}\cdot\vec{m} + \vec{\nu}\cdot\vec{n}} \tau_{\vec{m}\vec{n}},
%  \label{EigenChoi}
% \end{equation}
% % 
% 
% 
% 
% 
% 
% where $\tau_{\vec{m}\vec{n}}$ are the channel coefficients, the dimension of the ``vector indices" is $N$ and each vector element may attain the values $0$ or $1$. For instance, for $N=1$ we have
% % 
% \begin{equation}
%  \lambda_{\mu\nu}=\sum_{m,n=0}^1(-1)^{\mu m + \nu n} \tau_{m,n}.
% \end{equation}
% % 
% % 
% \subsection{Analytical derivation of conditions for PCE channels}
% % 
% Is it possible to infer the whole set of conditions (rules) a PCE channel must satisfy from the general expression for the eigenvalues of the Choi matrix (Eq. \ref{EigenChoi})? Work in progress...
% 
% % 
% \subsection{Extremal channels}
% % 
% It can be shown that PCE channels correspond to the extremal points in the polytope of trace preserving channels and non-negative $\tau$ coefficients for the case $M=1$. For $N>1$, in addition to the set of PCE channels, there are other  ``extremal channels", it there anything special around them?


