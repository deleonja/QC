\section{Alejo's ideas}
% 
\subsection{Eigenvalues of the Choi matrix}
% 
Due to the fact that the Pauli matrices and the identity form a basis of the Hilber-Shmidt space, the state of a qubit may be written as
% 
\begin{equation}
 \hat{\rho}=\frac{1}{2}\sum_{j,k=0}^1r_{jk}\hat{v}_{jk},
\end{equation}
%
where $\hat{v}_{00}=\dsone$, $\hat{v}_{01}=\hat{\sigma}_x$, $\hat{v}_{10}=i\hat{\sigma}_y$ and $\hat{v}_{11}=\hat{\sigma}_z$.
% 
Rewriting $\hat{\rho}$ in ``vector" notation, we have
%
\begin{equation}
 \vec{\rho}=\frac{1}{2}\sum_{jk=0}^1r_{jk}\vec{v}_{jk},
\end{equation}
%
The action of a quantum operation $\hat{M}$ (in its matrix form), capable of modifying each one of the coefficients $r_{jk}$ independently may be written as
%
\begin{equation}
 \hat{M}\vec{\rho}=\frac{1}{2}\sum_{jk=0}^1\tau_{jk}r_{jk}\vec{v}_{jk}.
\end{equation}
%
In our particular case (PCE operations), $\tau_{jk}=\{0,1\}$. 

In order to determine whether any specific operation, characterized by the set of coefficients $\{\tau_{jk}\}$ corresponds to a quantum channel, we must ensure that the eigenvalues associated to the Choi matrix are non-negative. For this, let write $\hat{M}$ in the vector Pauli basis
%
\begin{equation}
 \hat{M}=\sum_{jk=0}^1\tau_{jk}~\vec{v}_{jk}\vec{v}_{jk}^\intercal.
 \label{EQA1}
\end{equation}
%
On the other hand, it is easy to show that the elements of the computational and Pauli bases are related by $\vec{u}=\hat{T}\vec{v}$
\begin{equation}
\vec{u}=\begin{pmatrix}
\hat{u}_{00} \\ \hat{u}_{01} \\ \hat{u}_{10} \\ \hat{u}_{11}
\end{pmatrix}=\frac{1}{2}\begin{pmatrix}
1 & 0 & 0 & 1\\
0 & 1 & 1 & 0\\
0 & 1 & -1 & 0\\
1 & 0 & 0 & -1
\end{pmatrix} \begin{pmatrix}
\hat{v}_{00} \\ \hat{v}_{01} \\ \hat{v}_{10} \\ \hat{v}_{11}
\end{pmatrix},
\end{equation}
% 
where $\hat{u}_{jk}=\dyad{j}{k}$. The inverse relation reads $\vec{v}=\hat{S}\vec{u}$
\begin{equation}
\vec{v}=\begin{pmatrix}
\hat{v}_{00} \\ \hat{v}_{01} \\ \hat{v}_{10} \\ \hat{v}_{11}
\end{pmatrix}=\begin{pmatrix}
1 & 0 & 0 & 1\\
0 & 1 & 1 & 0\\
0 & 1 & -1 & 0\\
1 & 0 & 0 & -1
\end{pmatrix} \begin{pmatrix}
\hat{u}_{00} \\ \hat{u}_{01} \\ \hat{u}_{10} \\ \hat{u}_{11}
\end{pmatrix}.
\end{equation}
% 
Thus, for this ``indexing choice" of the elements of the bases, the transformation matrices satisfy $\hat{S}=2\hat{T}$. Then one can relate the bases elements in a compact form as
% 
\begin{equation}
 \vec{v}_{jk}=\sum_{mn=0}^1S_{mn}^{(jk)}\vec{u}_{mn},
\end{equation}
% 
by substituting this in Eq. \ref{EQA1} we get $\hat{M}$ in the computational basis
% 
\begin{align}
 \hat{M}=&\sum_{jk=0}^1\tau_{jk}~\left(\sum_{mn=0}^1S_{mn}^{(jk)}\vec{u}_{mn} \right) \left(\sum_{pq=0}^1S_{jk}^{(pq)}\vec{u}^\intercal_{pq} \right),\\
 =&\sum_{mnpq=0}^1\left(\sum_{jk=0}^1 S_{mn}^{(jk)}S_{jk}^{(pq)}  \tau_{jk} \right) \vec{u}_{mn}\vec{u}^\intercal_{pq}.\\
 &\sum_{mnpq=0}^1 M_{\substack{mn\\pq}} ~\vec{u}_{mn}\vec{u}^\intercal_{pq}.
\end{align}
% 
In order to obtain the Choi matrix $\hat{M}^c$, we must perform the reshuffling operation on $\hat{M}$, i.e. $M_{\substack{mn\\pq}} \to M_{\substack{mp\\nq}}$. In this way, $\hat{M}^c$ holds

\begin{equation}
\hat{M}^c=\sum_{mnpq=0}^1\left(\sum_{jk=0}^1 S_{mp}^{(jk)}S_{jk}^{(nq)}  \tau_{jk} \right) \vec{u}_{mn}\vec{u}^\intercal_{pq}.
\end{equation}
%
Back to the Pauli basis, we have
%
\begin{equation}
 \hat{M}^c=\sum_{\mu\nu\alpha\beta=0}^1\left\{\sum_{jk=0}^1\left[\sum_{mnpq=0}^1S_{mp}^{(jk)}S_{jk}^{(nq)}T_{\mu\nu}^{(mn)}T_{pq}^{(\alpha\beta)}\right]\tau_{jk}\right\}\vec{v}_{\mu\nu}\vec{v}_{\alpha\beta}^\intercal.
\end{equation}
%
By taking into account the relation between the transformation matrices $T_{\mu\nu}^{(mn)}=\frac{1}{2}S_{\mu\nu}^{(mn)}$, $\hat{M}^c$ reads
%
\begin{equation}
 \hat{M}^c=\frac{1}{4}\sum_{\mu\nu\alpha\beta=0}^1\left\{\sum_{jk=0}^1\left[\sum_{mnpq=0}^1S_{mp}^{(jk)}S_{jk}^{(nq)}S_{\mu\nu}^{(mn)}S_{pq}^{(\alpha\beta)}\right]\tau_{jk}\right\}\vec{v}_{\mu\nu}\vec{v}_{\alpha\beta}^\intercal.
\end{equation}
%
By noticing that $S_{mn}^{(jk)}=(-1)^m\delta_{jm}\delta_{kn}+\delta_{j,m\oplus 1}\delta_{k,n\oplus 1}$, and after several steps it is possible to show that 
% 
\begin{equation}
\frac{1}{4}\sum_{jk=0}^1\left[\sum_{mnpq=0}^1S_{mp}^{(jk)}S_{jk}^{(nq)}S_{\mu\nu}^{(mn)}S_{pq}^{(\alpha\beta)}\right]\tau_{jk}=\frac{1}{2}\delta_{\nu\beta}\delta_{\mu\alpha}\sum_{jk=0}^1(-1)^{j\mu\oplus k\nu}\tau_{jk},
\label{EQAL2}
\end{equation}
% 
after substituting, the Choi matrix reduces to
%
\begin{align}
 \hat{M}^c=&\frac{1}{2}\sum_{\mu\nu\alpha\beta=0}^1\left\{\delta_{\nu\beta}\delta_{\mu\alpha}\sum_{jk=0}^1(-1)^{j\mu\oplus k\nu}\tau_{jk}\right\}\vec{v}_{\mu\nu}\vec{v}_{\alpha\beta}^\intercal,\\
 =&\frac{1}{2}\sum_{\mu\nu=0}^1\left\{\sum_{jk=0}^1(-1)^{j\mu\oplus k\nu}\tau_{jk}\right\}\vec{v}_{\mu\nu}\vec{v}_{\mu\nu}^\intercal.
\end{align}
%
As it can be seen, this is the diagonal form of the Choi matrix. Thus, its 4 eigenvalues may be written in a compact form as
%
\begin{equation}
\lambda_{\mu\nu}=\frac{1}{2}\sum_{jk=0}^1(-1)^{j\mu\oplus k\nu}\tau_{jk},
\end{equation}
%
where $\mu,\nu=\{0,1\}$. Note in addition that this is in agreement with results already known in literature.

By following the same steps we can obtain the generalization to the $N$-qubits case. First of all note that $\hat{M}$ may be written as
%
\begin{equation}
 \hat{M}=\sum_{\vec{k}\vec{l}}\tau_{\vec{k}\vec{l}}~\vec{v}_{\vec{k}\vec{l}}\vec{v}_{\vec{k}\vec{l}}^\intercal,
%  \label{EQA1}
\end{equation}
% 
where
% 
\begin{equation}
 \vec{v}_{\vec{k}\vec{l}}=\vec{v}_{k_1l_1}\otimes\dots\vec{v}_{k_Nl_N}=\bigotimes_{j=1}^N\vec{v}_{k_jl_j}.
%  \label{EQA1}
\end{equation}
% 
Let write $\hat{M}$ in the $N$-qubits computational basis
% 
\begin{align}
 \hat{M}=&\sum_{\vec{k}\vec{l}\vec{p}\vec{q}}\left\{\sum_{\vec{m}\vec{n}} \left[ \prod_{i=1}^N S_{k_il_i}^{(m_in_i)}S_{m_in_i}^{(p_iq_i)} \right] \tau_{\vec{m}\vec{n}} \right\} \bigotimes_{j=1}^N\vec{u}_{k_jl_j}\vec{u}_{p_jq_j}^\intercal,\\
 =&\sum_{\vec{k}\vec{l}\vec{p}\vec{q}} M_{\vec{k}\vec{l}\vec{p}\vec{q}} ~\bigotimes_{j=1}^N\vec{u}_{k_jl_j}\vec{u}_{p_jq_j}^\intercal.
\end{align}
% 
After reshuffling ($M_{\vec{k}\vec{l}\vec{p}\vec{q}}\to M_{\vec{k}\vec{p}\vec{l}\vec{q}}$) we can get the associated Choi matrix. Back to the Pauli basis, $\hat{M}^c$ holds
%
\begin{equation}
 \hat{M}^c=\frac{1}{4^N}\sum_{\vec{\mu}\vec{\nu}\vec{\alpha}\vec{\beta}}\left\{\sum_{\vec{m}\vec{n}}\left[\prod_{i=1}^N\left(\sum_{k_il_ip_iq_i=0}^1 S_{k_ip_i}^{(m_in_i)} S_{m_in_i}^{(l_iq_i)}S_{\mu_i\nu_i}^{(k_il_i)}S_{p_iq_i}^{(\alpha_i\beta_i)}\right)\right]\tau_{\vec{m}\vec{n}}\right\} \bigotimes_{j=1}^N\vec{v}_{\mu_j\nu_j}\vec{v}_{\alpha_j\beta_j}^\intercal.
\end{equation}
% 
Using the result in Eq. \ref{EQAL2} and after some steps it can be shown that
%
\begin{equation}
 \hat{M}^c=\sum_{\vec{\mu}\vec{\nu}}\left\{\frac{1}{2^N}\sum_{\vec{\alpha}\vec{\beta}}(-1)^{\vec{\mu}\cdot\vec{\alpha}\oplus \vec{\nu}\cdot\vec{\beta}}\tau_{\vec{\alpha}\vec{\beta}}\right\}\vec{v}_{\vec{\mu}\vec{\nu}}\vec{v}_{\vec{\mu}\vec{\nu}}^\intercal,
\end{equation}
% 
which is also the diagonal form of $\hat{M}^c$. In this way, the $4^N$ eigenvalues may be written compactly as
% 
\begin{equation}
 \lambda_{\vec{\mu}\vec{\nu}}=\frac{1}{2^N}\sum_{\vec{\alpha}\vec{\beta}}(-1)^{\vec{\mu}\cdot\vec{\alpha}\oplus \vec{\nu}\cdot\vec{\beta}}\tau_{\vec{\alpha}\vec{\beta}}.
\end{equation}
% 
From this we can evaluate whether the set of coefficients $\tau_{\vec{\alpha}\vec{\beta}}$ satisfy the condition to the specific operation be considered a quantum channel, i.e. $\lambda_{\vec{\mu}\vec{\nu}}\geq 0$, for all $\vec{\mu},\vec{\nu}$.
 
\subsection{Analytical derivation of conditions for PCE channels}
% 
Is it possible to infer the whole set of conditions (rules) a PCE channel must satisfy from the general expression for the eigenvalues of the Choi matrix (Eq. \ref{EigenChoi})? Work in progress...

% 
\subsection{Extremal channels}
% 
It can be shown that PCE channels correspond to the extremal points in the polytope of trace preserving channels and non-negative $\tau$ coefficients for the case $M=1$. For $N>1$, in addition to the set of PCE channels, there are other  ``extremal channels", it there anything special around them?


\section{Notes on Eigenvalues}
% 
We have already proved that the eigenvalues associated to the Choi matrix of an arbitrary Pauli map on $N$ qubits read
% 
\begin{equation}
 \lambda_{\vec{\mu}}=\frac{1}{2^N}\sum_{\vec{\alpha}}(-1)^{\vec{\mu}\cdot\vec{\alpha}}\tau_{\vec{\alpha}}, 
 \label{Eq1}
\end{equation}
% 
where the coefficients $\tau_{\vec{\alpha}}$ in our particular case (PCE operations) are either 0 or 1 (erase or preserve Pauli components), $\vec{\alpha}=(\alpha_1,\cdots,\alpha_{2N})$, $\vec{\mu}=(\mu_1,\cdots,\mu_{2N})$, $\alpha_j=0,1$ and $\mu_k=0,1$.

Our task is then to find what conditions the coefficients $\tau_{\vec{\alpha}}$ must satisfy in order to the specific operation be considered a quantum channel, i.e. $\lambda_{\vec{\mu}}\geq 0$, $\forall$ $\vec{\mu}$. The only constraint we impose is that the operations must preserve the trace of density matrices, i.e. $\tau_{\vec{0}}=1$.

First of all consider that the operation has $r$ non-null coefficients, denoted by $\tau_{\vec{\beta}_0},\dots,\tau_{\vec{\beta}_{r-1}}$ (where $\tau_{\vec{\beta}_0}=\tau_{\vec{0}}$). In this way, the expression for the eigenvalues (Eq. \ref{Eq1}) is reduced to
% 
\begin{equation}
 \lambda_{\vec{\mu}}=\frac{1}{2^N}\sum_{j=0}^{r-1}(-1)^{\vec{\mu}\cdot\vec{\beta}_j}.
 \label{Eq2}
 \end{equation}
% 
Thus a PCE operation preserving $r$ $\left(1\leq r \leq 4^N\right)$ Pauli components may be considered a quantum channel if the associated indices $\vec{\beta}_j$ satisfy the condition
% 
\begin{equation}
 1+(-1)^{\vec{\mu}\cdot\vec{\beta}_1}+\cdots+(-1)^{\vec{\mu}\cdot\vec{\beta}_{r-1}}\geq 0,
 \label{Eq3}
 \end{equation}
% 
for all $\vec{\mu}$. Note that the trivial case $r=1$, corresponds to the completely depolarizing channel. 

In order to the above requirement to be fulfilled for $r>1$, is easy to see that we need to have at least $\left[\frac{r+1}{2}\right]$ positive, or at most $\left[\frac{r}{2}\right]$ negative terms, where $\left[x\right]$ denotes the integer part of $x$. Moreover, a positive term is obtained if $\vec{\mu}\cdot\vec{\beta}_{j}\Mod 2=0$ and a negative one if $\vec{\mu}\cdot\vec{\beta}_{j}\Mod 2=1$. Thus Eq. \ref{Eq3} is satisfied whenever
% 
\begin{equation}
\sum_{j=1}^{r-1} \left\{ \vec{\mu}\cdot\vec{\beta}_{j}\Mod 2 \right\}\leq \left[\frac{r}{2}\right], ~~\forall ~ \vec{\mu}.
 \label{Eq4}
 \end{equation}
%  
In particular, it can be easily seen that for $r=2$, any choice of multi-index $\vec{\beta}_{1}$ satisfies the condition $\vec{\mu}\cdot\vec{\beta}_{1}\Mod 2 \leq 1$, and thus any PCE operation with two non-null coefficients $\tau_{\vec{\beta}_0}=1$ and $\tau_{\vec{\beta}_1}=1$ is a quantum channel. In this case it is easy to show that the amount of PCE channels of this kind is given by
% 
\begin{equation}
 \mathcal{N}^{(2)}_N=4^N-1.
\label{Eq10}
 \end{equation}

Define $\gamma_{\vec{\mu}, j} \equiv \vec{\mu}\cdot\vec{\beta}_{j}\Mod 2$, thus the condition in Eq. \ref{Eq4} holds
% 
\begin{equation}
\sum_{j=1}^{r-1} \gamma_{\vec{\mu}, j} \leq \left[r/2\right], ~~\forall ~ \vec{\mu}.
 \label{Eq6}
 \end{equation}

Let explore some more general cases in detail.


\subsection{PCE operations preserving 3 components}
% 
The condition for the case $r=3$ reads
% 
\begin{equation}
 \gamma_{\vec{\mu}, 1} +\gamma_{\vec{\mu}, 2}\leq 1, ~~\forall ~ \vec{\mu}.
\label{Eq5}
 \end{equation}
% 
It can be shown that given an arbitrary $\vec{\beta}_{1}$, the only multi-index $\vec{\beta}_{2}$ satisfying Eq. \ref{Eq5} corresponds to the trivial one $\vec{\beta}_{0}$, for all $\vec{\mu}$ and arbitrary $N$. Then no PCE operation preserving three components is quantum channel.

% How to prove this in general? (trivial for $N=1$).

\subsection{PCE operations preserving 4 components}
% 
For $r=4$ we have
% 
\begin{equation}
 \gamma_{\vec{\mu}, 1} + \gamma_{\vec{\mu}, 2} + \gamma_{\vec{\mu}, 3} \leq 2, ~~\forall ~ \vec{\mu}.
\label{Eq7}
 \end{equation}
%
The only way to ensure that the inequality is not violated is by fixing one of the coefficients $\gamma$ as a sum modulo 2 of the remaining two, e.g. $\gamma_{\vec{\mu}, 3}=\gamma_{\vec{\mu}, 1}\oplus \gamma_{\vec{\mu}, 2}$, for arbitrary and different multi-indices $\vec{\beta}_{1}$, $\vec{\beta}_{2}$. The condition above reduces to 
% 
\begin{equation}
 \gamma_{\vec{\mu}, 1} + \gamma_{\vec{\mu}, 2} + \gamma_{\vec{\mu}, 1}\oplus\gamma_{\vec{\mu}, 2} \leq 2.
\label{Eq8}
 \end{equation}
%
It is easy to check that this expression is always satisfied for all $\vec{\mu}$ and arbitrary $N$. 

In conclusion, any PCE operation with 4 non-null components, characterized by the set of multi-indices $\{\vec{\beta}_{0},\vec{\beta}_{1},\vec{\beta}_{2},\vec{\beta}_{1}\oplus\vec{\beta}_{2}\}$, for arbitrary and different $\vec{\beta}_{1}$, $\vec{\beta}_{2}$ is a quantum channel. 

Moreover, it is possible to show that by employing the above construction we can obtain the following amount of different quantum channels
% 
\begin{equation}
 \mathcal{N}^{(4)}_N=\frac{(4^N-2)(4^N-1)}{6}.
\label{Eq9}
 \end{equation}
% 
The numerical value of $\mathcal{N}^{(4)}_N$ is in agreement with previous numerical treatments up to $N=6$, indicating that the construction above may be sufficient to generate all PCE channels preserving 4 Pauli components.

\subsection{PCE operations preserving more than 4 components}
% 
For $r=5$ we have
% 
\begin{equation}
 \gamma_{\vec{\mu}, 1} + \gamma_{\vec{\mu}, 2} + \gamma_{\vec{\mu}, 3} + \gamma_{\vec{\mu}, 4}\leq 2, ~~\forall ~ \vec{\mu}.
\label{Eq11}
 \end{equation}
%  
We can follow a procedure analogous to the previous one, this time adding a coefficient $\gamma_{\vec{\mu}, 4}$, associated to an independent multi-index $\vec{\beta}_{4}$,
% 
\begin{equation}
 \gamma_{\vec{\mu}, 1} + \gamma_{\vec{\mu}, 2} + \gamma_{\vec{\mu}, 1}\oplus\gamma_{\vec{\mu}, 2} + \gamma_{\vec{\mu}, 4}\leq 2,  ~~\forall ~ \vec{\mu}.
\label{Eq12}
 \end{equation}
% 
Nevertheless, as in the case $r=3$, it can be proved that the only multi-index satisfying the previous condition for all $\vec{\mu}$ corresponds to $\vec{\beta}_{4}=\vec{\beta}_{0}$. Furthermore one can carry out the same procedure for $r=6$ and $r=7$, obtaining the same result.

For $r=8$, in analogy to the case $r=4$ the optimal arrangement of $\gamma$ coefficients in the inequality reads
% 
\begin{equation}
 \gamma_{\vec{\mu}, 1} + \gamma_{\vec{\mu}, 2} + \gamma_{\vec{\mu}, 1}\oplus\gamma_{\vec{\mu}, 2} +\gamma_{\vec{\mu}, 4} + \gamma_{\vec{\mu}, 4}\oplus\gamma_{\vec{\mu}, 1} + \gamma_{\vec{\mu}, 4}\oplus\gamma_{\vec{\mu}, 2} + \gamma_{\vec{\mu}, 4}\oplus\gamma_{\vec{\mu}, 1}\oplus\gamma_{\vec{\mu}, 2}\leq 4.
\label{Eq13}
 \end{equation}
%
Then, any PCE operation with 8 non-null components, characterized by the set of multi-indices $\{\vec{\beta}_{0},\vec{\beta}_{1},\vec{\beta}_{2},\vec{\beta}_{1}\oplus\vec{\beta}_{2},\vec{\beta}_{4},\vec{\beta}_{4}\oplus\vec{\beta}_{1},\vec{\beta}_{4}\oplus\vec{\beta}_{2},\vec{\beta}_{4}\oplus\vec{\beta}_{1}\oplus\vec{\beta}_{2}\}$, for arbitrary and different $\vec{\beta}_{1}$, $\vec{\beta}_{2}$, $\vec{\beta}_{4}$ is a quantum channel.

One can perform the same procedure, and the next point the inequalities are satisfied is $r=16$. The results are summarized in Table \ref{table1}.


\begin{table}[t]
\centering
\begin{tabular}{|c|c|c|c|}
\hline 
$\vec{\beta}$ & $r$ & $[r/2]$ &                  \\ \hline \hline
 \cellcolor{gray} $\vec{\beta}_{0}$ & \cellcolor{gray} 1 & \cellcolor{gray} 0 & \multirow{4}{*}{$N\geq 1$} \\ \cline{1-3}
\cellcolor{lightgray}\textcolor{red}{$\vec{\beta}_{1}$} & \cellcolor{lightgray} 2 & \cellcolor{lightgray}1 &                    \\ \cline{1-3}
\textcolor{red}{$\vec{\beta}_{2}$} &  &  &                    \\ %\hline
\cellcolor{lightgray} $\vec{\beta}_{1}\oplus\vec{\beta}_{2}$ & \cellcolor{lightgray}4 & \cellcolor{lightgray}2 &                    \\ \hline\hline
\textcolor{red}{$\vec{\beta}_{4}$} &  &  &  \multirow{12}{*}{$N\geq 2$}                      \\ %\hline
$\vec{\beta}_{4}\oplus\vec{\beta}_{1}$ &  &  &                    \\ %\hline
$\vec{\beta}_{4}\oplus\vec{\beta}_{2}$ &  &  &                    \\ %\hline
\cellcolor{lightgray} $\vec{\beta}_{4}\oplus\vec{\beta}_{1}\oplus\vec{\beta}_{2}$ & \cellcolor{lightgray} 8 & \cellcolor{lightgray} 4 &                    \\ \cline{1-3}
\textcolor{red}{$\vec{\beta}_{8}$} &  &  &                    \\ %\hline
$\vec{\beta}_{8}\oplus\vec{\beta}_{1}$ &  &  &                    \\ %\hline
$\vec{\beta}_{8}\oplus\vec{\beta}_{2}$ &  &  &                    \\ %\hline
$\vec{\beta}_{8}\oplus\vec{\beta}_{4}$ &  &  &                    \\ %\hline
$\vec{\beta}_{8}\oplus\vec{\beta}_{1}\oplus\vec{\beta}_{2}$ &  &  &                    \\ %\hline
$\vec{\beta}_{8}\oplus\vec{\beta}_{1}\oplus\vec{\beta}_{4}$ &  &  &                    \\ %\hline
$\vec{\beta}_{8}\oplus\vec{\beta}_{2}\oplus\vec{\beta}_{4}$ &  &  &                    \\ %\hline
\cellcolor{lightgray} $\vec{\beta}_{8}\oplus\vec{\beta}_{1}\oplus\vec{\beta}_{2}\oplus\vec{\beta}_{4}$ & \cellcolor{lightgray} 16 & \cellcolor{lightgray} 8 &                    \\ %\hline
\hline \hline
\textcolor{red}{$\vec{\beta}_{16}$} &  &  &   \multirow{6}{*}{$N\geq 3$}                 \\ %\cline{1-3} \cline{1-3}
\dots &  &  &                    \\ %\cline{1-3}
\cellcolor{lightgray} $ \vec{\beta}_{16}\oplus \cdots \oplus \vec{\beta}_{8}$& \cellcolor{lightgray} 32 & \cellcolor{lightgray} 16 &                    \\ \cline{1-3}
\textcolor{red}{$\vec{\beta}_{32}$} &  &  &                     \\ %\cline{1-3}
\dots & &  &                    \\ %\cline{1-3}
\cellcolor{lightgray} $ \vec{\beta}_{32}\oplus \cdots \oplus \vec{\beta}_{16}$& \cellcolor{lightgray} 64 & \cellcolor{lightgray} 32 &                    \\ \cline{1-3}
\hline
\end{tabular}
\caption{Summary of sets of multi-indices $\vec{\beta}$ leading to PCE channels. For a given a number of qubits $N$, it is possible to obtain PCE quantum channels  preserving $r$ Pauli components. The first column shows the corresponding sets of multi-indices satisfying Eq. \ref{Eq4}. Note that for fixed values of $N$ and $r$, all multi-indices may be generated from a minimal set of independent $\vec{\beta}$'s (red). Finally note that this minimum amount of independent multi-indices is equal to $2N$.}
\label{table1}
\end{table}

