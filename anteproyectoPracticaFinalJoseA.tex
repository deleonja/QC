\documentclass[11pt, spanish, letterpage]{article}
\usepackage[T1]{fontenc}
\usepackage[utf8]{inputenc}
\usepackage[letterpaper]{geometry}
\geometry{verbose,tmargin=2cm,bmargin=2.5cm,lmargin=2cm,rmargin=2cm}
%\pagestyle{plain}
\setlength{\parskip}{\baselineskip}%espacio entre parrafos
\setlength{\parindent}{0mm}
\usepackage{graphicx}
%\usepackage{setspace}
\usepackage{tabulary}
\usepackage{amsmath}
\usepackage{amsfonts}
\usepackage{amssymb}
\usepackage{amsthm}
\usepackage{physics}
\usepackage{wrapfig}% para colocar figuras en diferentes posiciones
\usepackage{bbold}
%-----estiliza cajas
\usepackage{fancybox}
%-----------------

\usepackage{lipsum}
\usepackage{babel}
\usepackage{multirow}
\usepackage{array}
\renewcommand{\baselinestretch}{1} % interlinado
\addto\shorthandsspanish{\spanishdeactivate{~<>}}
\date{}
\spanishdecimal{.}
\usepackage{multicol}%para escribir en muchas columnas
%para que no corte palabras
    \usepackage[none]{hyphenat}
\usepackage{times}
%\onehalfspacing

\usepackage{hyperref}
\usepackage{apacite}%bibliografia y citas en formato apa

%---ejercicios, problemas -teoremas
%---Problemas encerrados-Bonitos
\usepackage[framemethod=tikz]{mdframed}
\mdfsetup{skipabove=\topskip,skipbelow=\topskip}
\newcounter{problem}[section]
\newenvironment{problem}[1][]{%
%\stepcounter{problem}%
\ifstrempty{#1}%
{\mdfsetup{%
frametitle={%
	\tikz[baseline=(current bounding box.east),outer sep=0pt]
	\node[anchor=east,rectangle,fill=brown!50]
{\strut Problema~\theproblem};}}

}%
{\mdfsetup{%
frametitle={%
	\tikz[baseline=(current bounding box.east),outer sep=0pt]
	\node[anchor=east,rectangle,fill=brown!50]
{\strut Problema ~\theproblem:~#1};}}%

}%
\mdfsetup{innertopmargin=5pt,linecolor=black!50,%
	linewidth=2pt,topline=true,
	frametitleaboveskip=\dimexpr-\ht\strutbox\relax,}
	
\begin{mdframed}[]\relax%
}{\end{mdframed}}

\newenvironment{solution}% environment name
{\colorbox{gray}{~~\textbf{\textcolor{white}{Solución:}}~~}~~}%
{}
%-----end------------

%\newtheorem{example}{Ejemplo}[chapter]
%\newtheorem{ejercicio}{Ejercicio}[chapter]
%%---
\newcommand{\Ev}{\mathbf{E}}
\newcommand{\rv}{\mathbf{r}}
\newcommand{\ru}{\hat{\rv}}
\newcommand{\zu}{\hat{\mathbf{z}}}
\usepackage{tabulary}
%---paquetes para fisica
\usepackage{physics}%facilita la escritura de operadores usados en fisica
%-paquete para unidades en el sistema internacional
\usepackage[load=prefix, load=abbr, load=physical]{siunitx}
\newunit{\gram}{g }%gramos
\newunit{\velocity}{ \metre / \Sec }%unidades de velocidad sistema internacional
\newunit{\acceleration}{ \metre / \Sec^2 }%unidades de aceleracion sistema internacional
\newunit{\entropy}{ \joule / \kelvin }%unidades de entropia sisteme internacional
%--definiendo constantes fisicas en el SI
\newcommand{\accgravity}{9.8 \metre / \Sec^2}
%---diferencial inexacta
\newcommand{\dbar}{\mathchar'26\mkern-12mu d}

\oddsidemargin 0in
\textwidth 6.5in
\topmargin -0.5in
\textheight 8.5in

\begin{document}

\begin{titlepage} % Suppresses displaying the page number on the title page and the subsequent page counts as page 1                                  
        \newcommand{\HRule}{\rule{\linewidth}{0.5mm}} % Defines a new command for horizontal lines, change thickness here                             

        \center % Centre everything on the page                                                                                                       
        
        %------------------------------------------------                                                                                             
        %       Title                                                                                                                                 
        %------------------------------------------------                                                                                             
			
        \HRule\\[0.6cm]

        {\huge\bfseries Estudio de canales cuánticos de muchos qubits con correlaciones restringidas}\\[0.5cm] % Title of your document                                                 

        \HRule\\[2cm]
        
        %------------------------------------------------                                                                                             
        %       Author(s)                                                                                                                             
        %------------------------------------------------                                                                                             


		\Large{\textbf{José Alfredo de León Garrido}}\\ [2cm] % Your name                                                                                          

        %------------------------------------------------                                                                                             
        %       Headings                                                                                                                              
        %------------------------------------------------                                                                                             

        \textsc{\LARGE Universidad de San Carlos de Guatemala\\ Escuela de Ciencias Físicas y Matemáticas\\ Licenciatura en Física}\\[2cm]
        
        \textsc{\huge Anteproyecto}\\
        \textsc{\Large Año de prácticas}\\[2cm]
        
        \textsc{\Large Supervisado por: \textbf{Dr. Carlos Pineda (IF-UNAM) y\\M.Sc. Juan Diego Chang (ICFM-USAC)}}
                                                                                                              

        %------------------------------------------------                                                                                             
        %       Date                                                                                                                                  
        %------------------------------------------------                                                                                             
        \vfill\vfill\vfill % Position the date 3/4 down the remaining page
        \vfill\vfill\vfill

        {\large 13 de enero 2020} % Date, change the \today to a set date if you want to be precise                                                              

        %------------------------------------------------                                                                                             
        %       Logo                                                                                                                                  
        %------------------------------------------------                                                                                             

        %\vfill\vfill                                                                                                                                 
        %\includegraphics[width=0.2\textwidth]{placeholder.jpg}\\[1cm] % Include a department/university logo - this will require the graphicx packag\                                                                                                                                                  

        %----------------------------------------------------------------------------------------                                                     

        \vfill % Push the date up 1/4 of the remaining page                                                                                           

\end{titlepage}

\section{Descripción general de la Institución}
\subsection{Instituto de Ciencias Físicas y Matemáticas - Universidad de San Carlos de Guatemala (ICFM-USAC)}
El Instituto de Investigación de Ciencias Físicas y Matemáticas (ICFM) es la unidad de la Escuela de Ciencias Físicas y Matemáticas (ECFM) que promueve y realiza estudios avanzados en áreas cientícas, fundamentales y aplicadas, de las ciencias físicas y matemáticas. El ICFM se proyecta como una plataforma regional de excelencia dedicada a la investigación y difusión del conocimiento en física y matemática. Las principales líneas de trabajo del ICFM son:
\begin{itemize}
	\item La investigación en ciencia básica y aplicada.
	\item La promoción de la investigación en ciencia básica y aplicada en el ámbito universitario.
	\item La difusión y divulgación del conocimiento generado por la investigación en ciencias físicas y matemáticas. 
	\item La actualización de programas académicos de ciencias físicas y matemáticas.
\end{itemize}

\subsection{Instituto de Física - Universidad Nacional Autónoma de México (IF-UNAM)}
Creado en 1938, el Instituto de Física (IF) ha crecido y madurado como institución académica para convertirse en uno de los centros de investigación en física más importantes de México, con un sólido prestigio internacional. En el IF se realiza una parte muy significativa de la investigación en física que se lleva a cabo en México, y se cultivan la docencia y formación de recursos humanos como actividades fundamentales.

El Instituto de Física ha contribuido de manera notable al desarrollo de la Física en México lo cual se refleja la calidad de sus aportaciones científicas y en la publicación de cerca de 6100 artículos, la mayoría en revistas de circulación internacional, además de otros múltiples productos de investigación. Como resultado de lo anterior sus académicos han obtenido un gran número de premios y distinciones.


\section{Descripción del grupo de trabajo}
\subsection{Grupo de Información y Óptica Cuántica (GIOC-UNAM)}
El grupo de Información y Óptica Cuántica de la Universidad Nacional Autónoma de México (GOIC-UNAM) está compuesto por investigadores y estudiantes de IIMAS-UNAM, ICN-UNAM y IF-UNAM. 

Desde su fundación, GIOC empezó como una serie de reuniones regulares con doble propósito: presentar los avances principales de investigacion tanto al grupo como también a la comunidad internacional, y entender los aspectos básicos de la mecánica cuántica y de la óptica cuántica en un nivel de posgrado. 

Las líneas de investigación en las que trabaja el grupo son:
\begin{itemize}
	\item Sistemas de espín
	\item Matrices aleatorias en información cuántica
	\item Markovianidad en sistemas cuánticos
	\item Información cuántica relativista
	\item Dinámica de rango
\end{itemize}
\pagebreak
\section{Descripción General del Proyecto}
%\subsection{Qubits}
Un bit cuántico, o \textit{qubit}, es la unidad fundamental de información de la computación e información cuántica, de igual manera que el bit lo es para la computación e información clásica. Un \textit{qubit}, o bien, un sistema de \textit{qubits}, está completamente descrito por su operador de densidad $\rho \in \mathcal{M}^{(n)}$; el cuál es un operador positivo, Hermítico y con traza unitaria que actúa sobre un espacio de Hilbert de $n$ dimensiones. 

% Para describrir al estado de un sistema de qubits se utiliza el formalismo de la matriz de densidad $\rho$; formulación alternativa que es equivalente a la de vector de estado, pues los postulados de la mecánica cuántica pueden ser reescritos en el lenguaje de la matriz de densidad. 

%\subsection{Canales cuánticos}
Para describir la evolución de sistemas cuánticos abiertos una de las herramientas más convenientes a utilizar es el formalismo matemático de los canales cuánticos. Este formalismo tiene la ventaja de describir adecuadamente cambios de estado discretos, es decir, transformaciones entre un estado inicial $\rho$ y un estado final $\rho '$ sin referencia explícita al paso del tiempo. 

Un canal cuántico es una transformación lineal $\Phi : \mathcal{M}^{(n)} \to \mathcal{M}^{(n)}$ que debe cumplir con algunas condiciones específicas para que $\Phi$ represente una operación física. La transformación $\Phi$ debe de ser un mapeo afín. De esa manera, la matriz $\Phi$ debe de ser tal que $\rho$, y su imagen $\rho '$, compartan las tres propiedades que deben satisfacer para representar a un sistema físico. Adicionalmente, puesto que $\rho$ puede ser extendido por una \textit{ancilla} $\sigma$ a un estado de un sistema compuesto de mayor dimensión, los canales cuánticos deben cumplir con la condición de completa positividad. 

%\subsection{Matriz de Choi}
En la práctica, evaluar las condiciones que la matriz $\Phi$ debe satisfacer resulta ser computacionalmente complicado. Por ello, es conveniente definir a la matriz de Choi $D_{\Phi}$, relacionada a su respectiva matriz $\Phi$ mediante un procedimiento de \textit{reshuffle}. Además, el teorema de Choi establece la condición que $D_{\Phi}$ debe cumplir para que $\Phi$ sea una transformación completamente positiva. 

El proyecto consiste en estudiar canales cuánticos de muchos qubits con correlaciones restringidas. Se pretende explorar la construcción de las matrices que representan a aquellos canales cuánticos que 'borran' correlaciones de un sistema de $n$ qubits. Estas matrices son de tamaño $4^n$. Por esa razón, se utilizarán primero herramientas computacionales para construir y organizar los canales, según el número de qubits y correlaciones borradas del sistema. Con esto, se quiere intentar elucidar una expresión anlítica que generalice la forma de estos canales cuánticos.  


\section{Objetivos}
\subsection{Objetivo principal}
\begin{itemize}
	\item Entender los canales cuánticos que borran correlaciones arbitrarias de un sistema de $n$ qubits. 	
\end{itemize}

\subsection{Objetivos específicos}
\begin{itemize}
	\item Entender las transformaciones completamente positivas que preservan la traza. 
	\item Construir, por medio de herramientas computacionales, la representación matricial de canales cuánticos que borran correlaciones arbitrarias de un sistema de $n$ qubits. 
	\item Explorar una manera analítica de construir la representación matricial estos canales.
\end{itemize}

\section{Justificación del Proyecto}


\section{Metodología}


\section{Cronograma}


\end{document}