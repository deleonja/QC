%Si no esta usando el iop package
%%Si no esta usando el iop package
%%Si no esta usando el iop package
%%Si no esta usando el iop package
%\input{definiciones}
%Comandos mios generales
\newcommand{\fig}[1]{fig.\ref{#1}}
\newcommand{\eqn}[1]{eqn.\eqref{#1}}
\newcommand{\tab}[1]{table \ref{#1}}
\newcommand{\medline}{\begin{center}\line(1,0){470}\end{center}}
\newcommand{\suspensivos}{{\color{blue} (.....)}}
\newcommand{\anotacion}[1]{{\color{red} (.. #1 ...)}}
\newcommand{\ie}{\textit{i. e.}}
%Quantum Mechanics Commands
\newcommand{\ket}[1]{{\vert #1 \rangle}}
\newcommand{\bra}[1]{{\langle #1 \vert}}
\newcommand{\proj}[2]{{\vert #1 \rangle \langle #2 \vert}}
\newcommand{\NoM}{{\cal N}}
%Lugares
\newcommand{\unam}{Universidad Nacional Aut\'onoma de M\'exico, M\'exico, D.F., M\'exico}
\newcommand{\icf}{Instituto de Ciencias F\'{\i}sicas, \unam}
\newcommand{\ifunam}{Instituto de F\'{\i}sica, \unam}
\newcommand{\ifunamint}{Instituto de F\'{\i}´ısica, Universidad Nacional Aut\'onoma de M\'exico, M\'exico D.F. 01000, M\'exico}
\newcommand{\fcen}{Departamento de F\'isica ``J. J. Giambiagi'', FCEN, Universidad de Buenos Aires, 1428 Buenos Aires, Argentina}
%%%%%%%%%%% Comandos de Carlos %%%%%%%%%%%%%%%%%%%%%%%%
% enges
%%%%%%%%%%%%%
%%%%%%%%%%%
\newif\ifenglish
\newif\ifespanol
\newif\ifshort
\newif\iflong
\shorttrue
\englishtrue
\newcommand{\enges}[2]{\ifenglish #1 \fi \ifespanol #2 \fi}
%%%%%%
%%%%%

%Comandos mios generales
\newcommand{\fig}[1]{fig.\ref{#1}}
\newcommand{\eqn}[1]{eqn.\eqref{#1}}
\newcommand{\tab}[1]{table \ref{#1}}
\newcommand{\medline}{\begin{center}\line(1,0){470}\end{center}}
\newcommand{\suspensivos}{{\color{blue} (.....)}}
\newcommand{\anotacion}[1]{{\color{red} (.. #1 ...)}}
\newcommand{\ie}{\textit{i. e.}}
%Quantum Mechanics Commands
\newcommand{\ket}[1]{{\vert #1 \rangle}}
\newcommand{\bra}[1]{{\langle #1 \vert}}
\newcommand{\proj}[2]{{\vert #1 \rangle \langle #2 \vert}}
\newcommand{\NoM}{{\cal N}}
%Lugares
\newcommand{\unam}{Universidad Nacional Aut\'onoma de M\'exico, M\'exico, D.F., M\'exico}
\newcommand{\icf}{Instituto de Ciencias F\'{\i}sicas, \unam}
\newcommand{\ifunam}{Instituto de F\'{\i}sica, \unam}
\newcommand{\ifunamint}{Instituto de F\'{\i}´ısica, Universidad Nacional Aut\'onoma de M\'exico, M\'exico D.F. 01000, M\'exico}
\newcommand{\fcen}{Departamento de F\'isica ``J. J. Giambiagi'', FCEN, Universidad de Buenos Aires, 1428 Buenos Aires, Argentina}
%%%%%%%%%%% Comandos de Carlos %%%%%%%%%%%%%%%%%%%%%%%%
% enges
%%%%%%%%%%%%%
%%%%%%%%%%%
\newif\ifenglish
\newif\ifespanol
\newif\ifshort
\newif\iflong
\shorttrue
\englishtrue
\newcommand{\enges}[2]{\ifenglish #1 \fi \ifespanol #2 \fi}
%%%%%%
%%%%%

%Comandos mios generales
\newcommand{\fig}[1]{fig.\ref{#1}}
\newcommand{\eqn}[1]{eqn.\eqref{#1}}
\newcommand{\tab}[1]{table \ref{#1}}
\newcommand{\medline}{\begin{center}\line(1,0){470}\end{center}}
\newcommand{\suspensivos}{{\color{blue} (.....)}}
\newcommand{\anotacion}[1]{{\color{red} (.. #1 ...)}}
\newcommand{\ie}{\textit{i. e.}}
%Quantum Mechanics Commands
\newcommand{\ket}[1]{{\vert #1 \rangle}}
\newcommand{\bra}[1]{{\langle #1 \vert}}
\newcommand{\proj}[2]{{\vert #1 \rangle \langle #2 \vert}}
\newcommand{\NoM}{{\cal N}}
%Lugares
\newcommand{\unam}{Universidad Nacional Aut\'onoma de M\'exico, M\'exico, D.F., M\'exico}
\newcommand{\icf}{Instituto de Ciencias F\'{\i}sicas, \unam}
\newcommand{\ifunam}{Instituto de F\'{\i}sica, \unam}
\newcommand{\ifunamint}{Instituto de F\'{\i}´ısica, Universidad Nacional Aut\'onoma de M\'exico, M\'exico D.F. 01000, M\'exico}
\newcommand{\fcen}{Departamento de F\'isica ``J. J. Giambiagi'', FCEN, Universidad de Buenos Aires, 1428 Buenos Aires, Argentina}
%%%%%%%%%%% Comandos de Carlos %%%%%%%%%%%%%%%%%%%%%%%%
% enges
%%%%%%%%%%%%%
%%%%%%%%%%%
\newif\ifenglish
\newif\ifespanol
\newif\ifshort
\newif\iflong
\shorttrue
\englishtrue
\newcommand{\enges}[2]{\ifenglish #1 \fi \ifespanol #2 \fi}
%%%%%%
%%%%%

%Comandos mios generales
\newcommand{\fig}[1]{fig.\ref{#1}}
\newcommand{\eqn}[1]{eqn.\eqref{#1}}
\newcommand{\tab}[1]{table \ref{#1}}
\newcommand{\medline}{\begin{center}\line(1,0){470}\end{center}}
\newcommand{\suspensivos}{{\color{blue} (.....)}}
\newcommand{\anotacion}[1]{{\color{red} (.. #1 ...)}}
\newcommand{\ie}{\textit{i. e.}}
%Quantum Mechanics Commands
\newcommand{\ket}[1]{{\vert #1 \rangle}}
\newcommand{\bra}[1]{{\langle #1 \vert}}
\newcommand{\proj}[2]{{\vert #1 \rangle \langle #2 \vert}}
\newcommand{\NoM}{{\cal N}}
%Lugares
\newcommand{\unam}{Universidad Nacional Aut\'onoma de M\'exico, M\'exico, D.F., M\'exico}
\newcommand{\icf}{Instituto de Ciencias F\'{\i}sicas, \unam}
\newcommand{\ifunam}{Instituto de F\'{\i}sica, \unam}
\newcommand{\ifunamint}{Instituto de F\'{\i}´ısica, Universidad Nacional Aut\'onoma de M\'exico, M\'exico D.F. 01000, M\'exico}
\newcommand{\fcen}{Departamento de F\'isica ``J. J. Giambiagi'', FCEN, Universidad de Buenos Aires, 1428 Buenos Aires, Argentina}
%%%%%%%%%%% Comandos de Carlos %%%%%%%%%%%%%%%%%%%%%%%%
% enges
%%%%%%%%%%%%%
%%%%%%%%%%%
\newif\ifenglish
\newif\ifespanol
\newif\ifshort
\newif\iflong
\shorttrue
\englishtrue
\newcommand{\enges}[2]{\ifenglish #1 \fi \ifespanol #2 \fi}
%%%%%%
%%%%%
