\documentclass[11pt,letterpaper]{article}
\usepackage[utf8]{inputenc}
\usepackage{amsmath}
\usepackage{amsfonts}
\usepackage{amssymb}
\usepackage{graphicx}
\usepackage{bbold}
\usepackage{physics}
\usepackage[]{lineno}  \linenumbers
\setlength\linenumbersep{3pt}
\author{\textbf{J.A. de León}}
\title{\Large Apuntes sobre el artículo \\ \emph{"Geometry of generalized Pauli channels" \cite{siudzinska2020geometry}}}
\date{}
\begin{document}
\maketitle

El mapeo de Pauli generalizado en cualquier dimensión $d$ se construye de $N\leq d+1$ bases mutuamente imparciales de la siguiente manera: 
\begin{equation}
	\Lambda = p_{N+1}\mathbb{1}	 + \sum _{\alpha = 0}^{N}p_{\alpha} \Phi _{\alpha},
	\label{eq:Lambda}
\end{equation}
donde $\Phi_{\alpha}=\sum _{k=0}^{d-1}P_k^{(\alpha)}\rho P_k^{\alpha}$ (con $P_k$ los proyectores al subespacio propio de $\ket{\psi _k^{(\alpha)}}$).\\


Las $p_{\alpha}$ deben satisfacer las siguientes condiciones para ser canales cuánticos:
\begin{equation}
	\begin{cases}
	p_{N+1} + \sum _{\alpha=1}^{N}p_{\alpha} + p_0 = 1,\\
	d^2p_{N+1} + d\sum _{\alpha=1}^{N}p_{\alpha} + p_0 \geq 0,\\
	dp_{\alpha} + p_0 \geq 0,\\
	p_0 \geq 0.
	\end{cases}
	\label{eq:p_conditions}
\end{equation}

Para el caso de 1 qubit $d=2$ y $N=3$. Por consiguiente, el mapeo generalizado de Pauli es de la forma
\begin{equation}
	\Lambda = p_{4}\mathbb{1}	 + \sum _{\alpha = 0}^{3}p_{\alpha} \Phi _{\alpha}.
	\label{eq:Lambda_1qubit}
\end{equation} 


Se construyen los canales cuánticos clásicos a partir de la expresión
\begin{align}
	\Phi _{\alpha} = \sum_{k=0}^{d-1} P_k^{(\alpha)}\rho P_k^{(\alpha)},
\end{align}
donde $P_k^{(\alpha)} = \dyad{\psi_k^{(\alpha)}}{\psi_k^{(\alpha)}}$. Así se tienen las $\Phi _{\alpha}$:
\begin{equation}
\{ \Phi _{\alpha} \}= 
\begin{cases}
\Phi _0 = \mqty(\frac{1}{2} & 0 & 0 & \frac{1}{2}\\ 0 & 0 & 0 & 0\\ 0 & 0 & 0 & 0\\ \frac{1}{2} & 0 & 0 & \frac{1}{2}), \\
\\
\Phi _1 = \mqty(\frac{1}{2} & 0 & 0 & \frac{1}{2}\\ 0 & \frac{1}{2} & \frac{1}{2} & 0\\ 0 & \frac{1}{2} & \frac{1}{2} & 0\\ \frac{1}{2} & 0 & 0 & \frac{1}{2}), \\
\\
\Phi _2 = \mqty(\frac{1}{2} & 0 & 0 & \frac{1}{2}\\ 0 & \frac{1}{2} & -\frac{1}{2} & 0\\ 0 & -\frac{1}{2} & \frac{1}{2} & 0\\ \frac{1}{2} & 0 & 0 & \frac{1}{2}), \\
\\
\Phi _3 = \mqty(1 & 0 & 0 & 0\\ 0 & 0 & 0 & 0\\ 0 & 0 & 0 & 0\\ 0 & 0 & 0 & 1). \\
\end{cases}
\label{eq:Phi_alphas-1qubit}
\end{equation}

Notemos que las matrices que aparecen en cada uno de los términos de la ecuación \eqref{eq:Lambda_1qubit} son los canales cuánticos del tipo que hemos estado estudiando para 1 qubit. Por consiguiente, estos canales son de la forma de los mapeos generalizados de Pauli con $p_i=1$ y el resto $p_j=0$. Estos valores de $p$ cumplen con las condiciones enunciadas en la ecuación \eqref{eq:p_conditions}.


\bibliographystyle{unsrt}
\bibliography{references}
	
	
\end{document}
